% generated from JIRA project LVV
% using template at /usr/share/miniconda/envs/docsteady-env/lib/python3.7/site-packages/docsteady/templates/tpr.latex.jinja2.
% using docsteady version 2.2.4
% Please do not edit -- update information in Jira instead
\documentclass[DM,lsstdraft,STR,toc]{lsstdoc}
\usepackage{geometry}
\usepackage{longtable,booktabs}
\usepackage{enumitem}
\usepackage{arydshln}
\usepackage{attachfile}
\usepackage{array}
\usepackage{dashrule}

\newcolumntype{L}[1]{>{\raggedright\let\newline\\\arraybackslash\hspace{0pt}}p{#1}}

\input meta.tex

\newcommand{\attachmentsUrl}{https://github.com/\gitorg/\lsstDocType-\lsstDocNum/blob/\gitref/attachments}
\providecommand{\tightlist}{
  \setlength{\itemsep}{0pt}\setlength{\parskip}{0pt}}

\setcounter{tocdepth}{4}

\begin{document}

\def\milestoneName{Alignment System Verification}
\def\milestoneId{}
\def\product{Alignment System}

\setDocCompact{true}

\title{LVV-P84: Alignment System Verification Test Plan and Report}
\setDocRef{\lsstDocType-\lsstDocNum}
\date{ 2022-03-08 }
\author{ Sandrine Thomas }

% Most recent last
\setDocChangeRecord{
\addtohist{}{2022-03-04}{First draft}{Sandrine Thomas}
}

\setDocCurator{Sandrine Thomas}
\setDocUpstreamLocation{\url{https://github.com/lsst-dm/\lsstDocType-\lsstDocNum}}
\setDocUpstreamVersion{\vcsRevision}



\setDocAbstract{
This is the test plan and report for
\textbf{ Alignment System Verification},
an LSST milestone pertaining to the Data Management Subsystem.\\
This document is based on content automatically extracted from the Jira test database on \docDate.
The most recent change to the document repository was on \vcsDate.
}


\maketitle

\section{Introduction}
\label{sect:intro}


\subsection{Objectives}
\label{sect:objectives}

 {The objective of this test plan is to verify the Spatial Analyzer API
is working before shipping the laser tracker from Tucson to Chile. This
test plan consists of one test cycle that was initially started in March
2020 and was continued in July 2021. }



\subsection{System Overview}
\label{sect:systemoverview}

 {The Spatial Analyzer Controller (T2SA) interfaces with the laser
tracker through the Spatial Analyzer API to execute scripts and measure
the positions of the spherically mounted retroreflectors (SMR's).}


\subsection{Document Overview}
\label{sect:docoverview}

This document was generated from Jira, obtaining the relevant information from the
\href{https://jira.lsstcorp.org/secure/Tests.jspa\#/testPlan/LVV-P84}{LVV-P84}
~Jira Test Plan and related Test Cycles (
\href{https://jira.lsstcorp.org/secure/Tests.jspa\#/testCycle/LVV-C150}{LVV-C150}
).

Section \ref{sect:intro} provides an overview of the test campaign, the system under test (\product{}),
the applicable documentation, and explains how this document is organized.
Section \ref{sect:testplan} provides additional information about the test plan, like for example the configuration
used for this test or related documentation.
Section \ref{sect:personnel} describes the necessary roles and lists the individuals assigned to them.

Section \ref{sect:overview} provides a summary of the test results, including an overview in Table \ref{table:summary},
an overall assessment statement and suggestions for possible improvements.
Section \ref{sect:detailedtestresults} provides detailed results for each step in each test case.

The current status of test plan \href{https://jira.lsstcorp.org/secure/Tests.jspa\#/testPlan/LVV-P84}{LVV-P84} in Jira is \textbf{ Completed }.

\subsection{References}
\label{sect:references}
\renewcommand{\refname}{}
\bibliography{lsst,refs,books,refs_ads,local}


\newpage
\section{Test Plan Details}
\label{sect:testplan}


\subsection{Data Collection}

  Observing is not required for this test campaign.

\subsection{Verification Environment}
\label{sect:hwconf}
  {The SA API will be verified in the laser lab in Tucson.}

  \subsection{Entry Criteria}
  {In order to run these tests, the following criteria must be met first:}

\begin{itemize}
\tightlist
\item
  {All communications between the MTAlignment CSC and T2SA have already
  been verified}
\end{itemize}

  \subsection{Exit Criteria}
  {In order for this event to be considered complete, the following
criteria must be met:\\
}

\begin{itemize}
\tightlist
\item
  {All of the executions of the test cases have been populated with the
  actual results of the tests}
\end{itemize}


\subsection{Related Documentation}


\begin{longtable}{rp{10cm}l}
\multicolumn{3}{c}{Jira Attachments} \\ \hline
 To LVV-C150 results &
  AutoVectorsGroupsCAM\_VertTrackertoCAM\_HorizontalTracker.pdf & \attachfile{attachments/AutoVectorsGroupsCAM_VertTrackertoCAM_HorizontalTracker.pdf}\\ \hline
To LVV-C150 results &
  T2SATunnelTemplateMarch\_12\_Hor2020\_03\_12\_17\_26.xit64 & \attachfile{attachments/T2SATunnelTemplateMarch_12_Hor2020_03_12_17_26.xit64}\\ \hline
        \end{longtable}

All documents provided as attachments in Jira are downloaded to Github and linked here for convenience.
However, since they are not properly versioned, they should be considered informal and therefore
not be part of the verification baseline.


\subsection{PMCS Activity}

Primavera milestones related to the test campaign:
Alignment July Acceptance Test
-~\href{https://jira.lsstcorp.org/browse/DM-31263}{DM-31263}


\newpage
\section{Personnel}
\label{sect:personnel}

The personnel involved in the test campaign is shown in the following table.

{\small
\begin{longtable}{p{3cm}p{3cm}p{3cm}p{6cm}}
\hline
\multicolumn{2}{r}{T. Plan \href{https://jira.lsstcorp.org/secure/Tests.jspa\#/testPlan/LVV-P84}{LVV-P84} owner:} &
\multicolumn{2}{l}{\textbf{ Sandrine Thomas } }\\\hline
\multicolumn{2}{r}{T. Cycle \href{https://jira.lsstcorp.org/secure/Tests.jspa\#/testCycle/LVV-C150}{LVV-C150} owner:} &
\multicolumn{2}{l}{\textbf{
Sandrine Thomas }
} \\\hline
\textbf{Test Cases} & \textbf{Assigned to} & \textbf{Executed by} & \textbf{Additional Test Personnel} \\ \hline
\href{https://jira.lsstcorp.org/secure/Tests.jspa#/testCase/LVV-T1816}{LVV-T1816}
& {\small Sandrine Thomas } & {\small  } &
\begin{minipage}[]{6cm}
\smallskip
{\small New River Kinematics Personnel\\
Rubin Observatory Software Developer\\
T\&S Scientist }
\medskip
\end{minipage}
\\ \hline
\href{https://jira.lsstcorp.org/secure/Tests.jspa#/testCase/LVV-T1814}{LVV-T1814}
& {\small Sandrine Thomas } & {\small Colin Winslow } &
\begin{minipage}[]{6cm}
\smallskip
{\small New River Kinematics Personnel\\
Rubin Observatory Software Developer\\
T\&S Scientist }
\medskip
\end{minipage}
\\ \hline
\href{https://jira.lsstcorp.org/secure/Tests.jspa#/testCase/LVV-T1817}{LVV-T1817}
& {\small Sandrine Thomas } & {\small Colin Winslow } &
\begin{minipage}[]{6cm}
\smallskip
{\small New River Kinematics Personnel\\
Rubin Observatory Software Developer\\
T\&S Scientist }
\medskip
\end{minipage}
\\ \hline
\href{https://jira.lsstcorp.org/secure/Tests.jspa#/testCase/LVV-T1813}{LVV-T1813}
& {\small Sandrine Thomas } & {\small Mostafa Lutfi } &
\begin{minipage}[]{6cm}
\smallskip
{\small New River Kinematics Personnel\\
Rubin Observatory Software Developer\\
T\&S Scientist }
\medskip
\end{minipage}
\\ \hline
\href{https://jira.lsstcorp.org/secure/Tests.jspa#/testCase/LVV-T1815}{LVV-T1815}
& {\small Sandrine Thomas } & {\small Colin Winslow } &
\begin{minipage}[]{6cm}
\smallskip
{\small New River Kinematics Personnel\\
Rubin Observatory Software Developer\\
T\&S Scientist }
\medskip
\end{minipage}
\\ \hline
\href{https://jira.lsstcorp.org/secure/Tests.jspa#/testCase/LVV-T2181}{LVV-T2181}
& {\small Sandrine Thomas } & {\small Colin Winslow } &
\begin{minipage}[]{6cm}
\smallskip
{\small New River Kinematics Personnel\\
Rubin Observatory Software Developer\\
T\&S Scientist }
\medskip
\end{minipage}
\\ \hline
\end{longtable}
}

\newpage

\section{Test Campaign Overview}
\label{sect:overview}

\subsection{Summary}
\label{sect:summarytable}

{\small
\begin{longtable}{p{2cm}cp{2.3cm}p{8.6cm}p{2.3cm}}
\toprule
\multicolumn{2}{r}{ T. Plan \href{https://jira.lsstcorp.org/secure/Tests.jspa\#/testPlan/LVV-P84}{LVV-P84}:} &
\multicolumn{2}{p{10.9cm}}{\textbf{ Alignment System Verification }} & Completed \\\hline
\multicolumn{2}{r}{ T. Cycle \href{https://jira.lsstcorp.org/secure/Tests.jspa\#/testCycle/LVV-C150}{LVV-C150}:} &
\multicolumn{2}{p{10.9cm}}{\textbf{ T2SA Factory Acceptance Test }} & Done \\\hline
\textbf{Test Cases} &  \textbf{Ver.} & \textbf{Status} & \textbf{Comment} & \textbf{Issues} \\\toprule
\href{https://jira.lsstcorp.org/secure/Tests.jspa#/testCase/LVV-T1816}{LVV-T1816}
&  1
& Initial Pass &
\begin{minipage}[]{9cm}
\smallskip
In July 2021 this execution was set to fail as the purpose was to verify
that the laser was behaving when the laser was aligned with the optical
axis of the optics. We realized that there was a problem of reflectors
visibility and therefore had to make a mechanical change to how we will
mount the laser in the cell. The mount will be at 45 deg and the tests
will be repeated with the nominal position being tilted.
\medskip
\end{minipage}
&   \\\hline
\href{https://jira.lsstcorp.org/secure/Tests.jspa#/testCase/LVV-T1814}{LVV-T1814}
&  1
& Fail &
\begin{minipage}[]{9cm}
\smallskip
The results of this test execution were captured while the laser tracker
was set up in a vertical position. These steps were repeated when the
laser tracker was reconfigured with a 45degree tilt in the \emph{Testing
of the T2SA with the laser tracker at 45 degrees} test case.~
\medskip
\end{minipage}
&   \href{https://jira.lsstcorp.org/browse/LVV-19977}{LVV-19977}
\href{https://jira.lsstcorp.org/browse/LVV-19978}{LVV-19978}
\href{https://jira.lsstcorp.org/browse/LVV-19979}{LVV-19979}
\href{https://jira.lsstcorp.org/browse/LVV-19977}{LVV-19977}
\href{https://jira.lsstcorp.org/browse/LVV-19978}{LVV-19978}
\href{https://jira.lsstcorp.org/browse/LVV-19979}{LVV-19979}
\\\hline
\href{https://jira.lsstcorp.org/secure/Tests.jspa#/testCase/LVV-T1817}{LVV-T1817}
&  1
& Pass &
\begin{minipage}[]{9cm}
\smallskip
The results of this test execution were captured while the laser tracker
was set up in a vertical position. These steps were repeated when the
laser tracker was reconfigured with a 45degree tilt in the \emph{Testing
of the T2SA with the laser tracker at 45 degrees} test case.
\medskip
\end{minipage}
&   \\\hline
\href{https://jira.lsstcorp.org/secure/Tests.jspa#/testCase/LVV-T1813}{LVV-T1813}
&  1
& Fail &
\begin{minipage}[]{9cm}
\smallskip
The results of this test execution were captured during the continuation
of the tests in July 2021 after the laser tracker was reconfigured with
a 45 degree tilt.
\medskip
\end{minipage}
&   \href{https://jira.lsstcorp.org/browse/LVV-19986}{LVV-19986}
\href{https://jira.lsstcorp.org/browse/LVV-19987}{LVV-19987}
\href{https://jira.lsstcorp.org/browse/LVV-19988}{LVV-19988}
\href{https://jira.lsstcorp.org/browse/LVV-19989}{LVV-19989}
\href{https://jira.lsstcorp.org/browse/LVV-19986}{LVV-19986}
\href{https://jira.lsstcorp.org/browse/LVV-19987}{LVV-19987}
\href{https://jira.lsstcorp.org/browse/LVV-19988}{LVV-19988}
\href{https://jira.lsstcorp.org/browse/LVV-19989}{LVV-19989}
\\\hline
\href{https://jira.lsstcorp.org/secure/Tests.jspa#/testCase/LVV-T1815}{LVV-T1815}
&  1
& Pass &
\begin{minipage}[]{9cm}
\smallskip
The results of this test execution were captured during the continuation
of the tests in July 2021 after the laser tracker was reconfigured with
a 45 degree tilt.
\medskip
\end{minipage}
&   \\\hline
\href{https://jira.lsstcorp.org/secure/Tests.jspa#/testCase/LVV-T2181}{LVV-T2181}
&  1
& Blocked &
\begin{minipage}[]{9cm}
\smallskip
As mentioned by the objective, this test case is meant to redo some of
the original testing that was done when the laser tracker was set up
horizontally. Specifically, the~\emph{Position Measurement of M1M3, M2
and the camera~}test case and the \emph{Motion tests} test case have
been called to test and will be repeated with the laser tracker tilted
at 45deg.
\medskip
\end{minipage}
&   \href{https://jira.lsstcorp.org/browse/LVV-19978}{LVV-19978}
\href{https://jira.lsstcorp.org/browse/LVV-19978}{LVV-19978}
\\\hline
\caption{Test Campaign Summary}
\label{table:summary}
\end{longtable}
}

\subsection{Overall Assessment}
\label{sect:overallassessment}

{As a result of the Factory acceptance testing, the overall test can be
considered a PASS. The test cycle was updated with additional tests that
were a subset of tests that were initially done in March 2020. The tests
that were done in July 2021 were able to address some of the problems
that were seen as part of these initial tests and were seen to be passed
as part of the latest test event. }

\subsection{Recommended Improvements}
\label{sect:recommendations}

Not yet available.

\newpage
\section{Detailed Test Results}
\label{sect:detailedtestresults}

\subsection{Test Cycle LVV-C150 }

Open test cycle {\it \href{https://jira.lsstcorp.org/secure/Tests.jspa#/testrun/LVV-C150}{T2SA Factory Acceptance Test}} in Jira.

Test Cycle name: T2SA Factory Acceptance Test\\
Status: Done

This covers the verification of the interface between the Spatial
Analyzer software that controls the laser tracker and the Commandable
SAL Component called the Alignment System Controller.\\
This is executed by the vendor, New River Kinematics, with our
assistance.

\subsubsection{Software Version/Baseline}
Not provided.

\subsubsection{Configuration}
{In March 2020, the T2SA Acceptance tests were initially executed with
the laser tracker in a vertical position. The initial tests were put on
hold due to the pandemic so the tests were continued in July 2021.
However, when the tests were continued in July, it was discovered that
the laser had a hard time reaching the retroreflector on the camera
while parallel to the optical axis of M1M3. As a result, the laser
tracker was reconfigured with a 45 deg tilt and a subset of the initial
verification tests that were done in March 2020 were repeated in July
2021 with the new configuration. }

\subsubsection{Test Cases in LVV-C150 Test Cycle}

\paragraph{ LVV-T1816 - Measurement with the laser set up horizontally }\mbox{}\\

Version \textbf{1}.
Open  \href{https://jira.lsstcorp.org/secure/Tests.jspa#/testCase/LVV-T1816}{\textit{ LVV-T1816 } }
test case in Jira.

The objective of this test will be to verify the calibration of the
laser holds whenever the orientation changes. Because the laser will be
part of the M1M3 mirror cell, it is expected to change orientation
several times during the night. This is mostly important for the tests
that we want to conduct during the commissioning phase.

\textbf{ Preconditions}:\\
The commands and interfaces have been successfully tested with the laser
in a vertical position before being tested horizontally.

Execution status: {\bf Initial Pass }

Final comment:\\In July 2021 this execution was set to fail as the purpose was to verify
that the laser was behaving when the laser was aligned with the optical
axis of the optics. We realized that there was a problem of reflectors
visibility and therefore had to make a mechanical change to how we will
mount the laser in the cell. The mount will be at 45 deg and the tests
will be repeated with the nominal position being tilted.


Detailed steps results:

\begin{tabular}{p{2cm}p{14cm}}
\toprule
Step 1 & Step Execution Status: \textbf{ Pass } \\ \hline
\end{tabular}
 Description \\
{\footnotesize
Load Compensation using LOAD\_TRACKER\_COMPENSATION\\
If the file does not exist the error is 335

}
\hdashrule[0.5ex]{\textwidth}{1pt}{3mm}
  Expected Result \\
{\footnotesize
The files are loaded.

}
\hdashrule[0.5ex]{\textwidth}{1pt}{3mm}
  Actual Result \\
{\footnotesize
Verified files load, and incorrectly named files produce error 335

}
\begin{tabular}{p{2cm}p{14cm}}
\toprule
Step 2 & Step Execution Status: \textbf{ Pass } \\ \hline
\end{tabular}
 Description \\
{\footnotesize
Measure M1M3, M2 and the camera position.\\
Ask for POS and OFFSET~

}
\hdashrule[0.5ex]{\textwidth}{1pt}{3mm}
  Expected Result \\
{\footnotesize
The measurements for the positions and offsets of the M1M3, M2 and
Camera are shown.

}
\hdashrule[0.5ex]{\textwidth}{1pt}{3mm}
  Actual Result \\
{\footnotesize
\begin{verbatim}
Forgot to query positions... whoops!

Offsets:
'ACK-300 RefFrame:FrameCAM_0.00_0.00_0.00_12;X:8.958377;Y:75.981318;Z:-664.202237;Rx:3.287680;Ry:-0.617449;Rz:1.037039;03/12/2020 15:39:54\r\n'
\end{verbatim}

\begin{verbatim}
'ACK-300 RefFrame:FrameM2_0.00_0.00_0.00_12;X:-19.366712;Y:-5.084746;Z:-1751.262655;Rx:-0.839965;Ry:0.386325;Rz:1.880290;03/12/2020 15:39:55\r\n'
\end{verbatim}

\begin{verbatim}
'ACK-300 RefFrame:FrameM1M3_0.00_0.00_0.00_12;X:0.001793;Y:0.000320;Z:-0.002017;Rx:0.000057;Ry:-0.000031;Rz:-0.000012;03/12/2020 15:39:56\r\n'
\end{verbatim}

}
\begin{tabular}{p{2cm}p{14cm}}
\toprule
Step 3 & Step Execution Status: \textbf{ Pass } \\ \hline
\end{tabular}
 Description \\
{\footnotesize
Move the laser horizontaly and set the altitude to the following:\\
For this step Alt = 0, Az = 0 and rot = 0

}
\hdashrule[0.5ex]{\textwidth}{1pt}{3mm}
  Expected Result \\
{\footnotesize
The laser is moved and the altitude is set.~

}
\hdashrule[0.5ex]{\textwidth}{1pt}{3mm}
  Actual Result \\
{\footnotesize
This works fine.~

}
\begin{tabular}{p{2cm}p{14cm}}
\toprule
Step 4 & Step Execution Status: \textbf{ Not Executed } \\ \hline
\end{tabular}
 Description \\
{\footnotesize
Redo the compensation. LOAD\_TRACKER\_COMPENSATION

}
\hdashrule[0.5ex]{\textwidth}{1pt}{3mm}
  Expected Result \\
{\footnotesize
The files are loaded.~

}
\hdashrule[0.5ex]{\textwidth}{1pt}{3mm}
  Actual Result \\
{\footnotesize
This step was not executed because it was determined that it wasn't
necessary for this test.

}
\begin{tabular}{p{2cm}p{14cm}}
\toprule
Step 5 & Step Execution Status: \textbf{ Initial Pass } \\ \hline
\end{tabular}
 Description \\
{\footnotesize
Remeasure M1M3, M2 and Camera (CMD: )\\
Ask for POS and OFFSET

}
\hdashrule[0.5ex]{\textwidth}{1pt}{3mm}
  Expected Result \\
{\footnotesize
The measurements for the positions and offsets of the M1M3, M2 and
Camera are shown.

}
\hdashrule[0.5ex]{\textwidth}{1pt}{3mm}
  Actual Result \\
{\footnotesize
Measured M1m3 sucessfully.\\
Camera measurement failed least squares fit because it's too tight of an
angle for the tracker.~\\[2\baselineskip]Unable to track some SMRs
because of the handle on the tracker blocking the beam, so we moved the
cart closer to get more favorable angles.\\[2\baselineskip]We remeasured
everything, but since we moved the cart between measuring with the
tracker in a vertical orientation and remeasuring in horizontal
orientation, we unfortunately can't make a direct
comparison.\\[2\baselineskip]SO instead, we compared the measured
distances between each or the 3 ``Camera'' SMRs with the tracker in both
orientations. The difference between the two measurements is \textless{}
4 microns.\\[2\baselineskip]Generated a SA report detailing these
results

}

\paragraph{ LVV-T1814 - Position Measurement of M1M3, M2 and the camera }\mbox{}\\

Version \textbf{1}.
Open  \href{https://jira.lsstcorp.org/secure/Tests.jspa#/testCase/LVV-T1814}{\textit{ LVV-T1814 } }
test case in Jira.

The main objective of this test is to verify the T2SA is able to measure
the positions of the fiducials placed adjacent to Camera and M1M3 using
the Laser Tracker.

\textbf{ Preconditions}:\\
The set-up procedure and all communications between the MTAlignment CSC
and T2SA were verified.\\[2\baselineskip]

Execution status: {\bf Fail }

Final comment:\\The results of this test execution were captured while the laser tracker
was set up in a vertical position. These steps were repeated when the
laser tracker was reconfigured with a 45degree tilt in the \emph{Testing
of the T2SA with the laser tracker at 45 degrees} test case.~

Issues found during the execution of LVV-T1814 test case:

\begin{itemize}
\item \href{https://jira.lsstcorp.org/browse/LVV-19977}{LVV-19977}~~Blocked SMR Coding Workaround

\item \href{https://jira.lsstcorp.org/browse/LVV-19978}{LVV-19978}~~Update T2SA Test Steps

\item \href{https://jira.lsstcorp.org/browse/LVV-19979}{LVV-19979}~~T2SA SAVE\_SA\_JOBFILE Command

\end{itemize}

Detailed steps results:

\begin{tabular}{p{2cm}p{14cm}}
\toprule
Step 1 & Step Execution Status: \textbf{ Pass w/ Deviation } \\ \hline
\end{tabular}
 Description \\
{\footnotesize
To start a measurement, one needs to set the reference group of point
using the command SET\_REFERENCE\_GROUP. LTS-966-REQ-0015\\
Start without specifying a group name to trigger the
error\\[2\baselineskip]

}
\hdashrule[0.5ex]{\textwidth}{1pt}{3mm}
  Expected Result \\
{\footnotesize
Error 313 should be triggered

}
\hdashrule[0.5ex]{\textwidth}{1pt}{3mm}
  Actual Result \\
{\footnotesize
Expected Error 313 but got~

\begin{verbatim}
'ERR-306: Did not find or set point group and target name.\r\n'
\end{verbatim}

}
\begin{tabular}{p{2cm}p{14cm}}
\toprule
Step 2 & Step Execution Status: \textbf{ Pass } \\ \hline
\end{tabular}
 Description \\
{\footnotesize
Repeat the step above with a valid group name. LTS-966-REQ-0015

}
\hdashrule[0.5ex]{\textwidth}{1pt}{3mm}
  Expected Result \\
{\footnotesize
​​​​Verify in SA that the group name is the correct one

}
\hdashrule[0.5ex]{\textwidth}{1pt}{3mm}
  Actual Result \\
{\footnotesize
Set Reference Group to M2, verified this shows up in SA

}
\begin{tabular}{p{2cm}p{14cm}}
\toprule
Step 3 & Step Execution Status: \textbf{ Pass } \\ \hline
\end{tabular}
 Description \\
{\footnotesize
Next, one needs to set the working frame using SET\_WORKING\_FRAME.
Start without specifying a group name to trigger the error
LTS-966-REQ-0018

}
\hdashrule[0.5ex]{\textwidth}{1pt}{3mm}
  Expected Result \\
{\footnotesize
Error 314 should be triggered

}
\hdashrule[0.5ex]{\textwidth}{1pt}{3mm}
  Actual Result \\
{\footnotesize
Indeed we get Error 314

}
\begin{tabular}{p{2cm}p{14cm}}
\toprule
Step 4 & Step Execution Status: \textbf{ Pass } \\ \hline
\end{tabular}
 Description \\
{\footnotesize
Repeat the test from above giving a working frame with the right format~

}
\hdashrule[0.5ex]{\textwidth}{1pt}{3mm}
  Expected Result \\
{\footnotesize
Verify in SA that the working frame is the correct one. ACK 300

}
\hdashrule[0.5ex]{\textwidth}{1pt}{3mm}
  Actual Result \\
{\footnotesize
It works. As a naming convention for frames in our template files, we
prefix with the word FRAME, ie FRAMEM2, FRAMECAM, etc

}
\begin{tabular}{p{2cm}p{14cm}}
\toprule
Step 5 & Step Execution Status: \textbf{ Not Executed } \\ \hline
\end{tabular}
 Description \\
{\footnotesize
Use the command SET\_STATION\_LOCK with true and check that the laser is
still locked on a SMR even when we move them LTS-966-REQ-0020

}
\hdashrule[0.5ex]{\textwidth}{1pt}{3mm}
  Expected Result \\
{\footnotesize
The laser should follow the target

}
\hdashrule[0.5ex]{\textwidth}{1pt}{3mm}
  Actual Result \\
{\footnotesize
There is a misunderstanding of what the SET\_STATION\_LOCK command is
expected to do. This test step was skipped because the expected result
is unrelated to what the step specifies.~

}
\begin{tabular}{p{2cm}p{14cm}}
\toprule
Step 6 & Step Execution Status: \textbf{ Not Executed } \\ \hline
\end{tabular}
 Description \\
{\footnotesize
Check of error 331, fail to lock station ?

}
\hdashrule[0.5ex]{\textwidth}{1pt}{3mm}
  Expected Result \\
{\footnotesize
There is no error as a result of the SET\_STATION\_LOCK command.~

}
\hdashrule[0.5ex]{\textwidth}{1pt}{3mm}
  Actual Result \\
{\footnotesize
There is a misunderstanding of what the SET\_STATION\_LOCK command is
expected to do. This test step was skipped because the expected error is
unrelated to what the command actually does.~

}
\begin{tabular}{p{2cm}p{14cm}}
\toprule
Step 7 & Step Execution Status: \textbf{ Not Executed } \\ \hline
\end{tabular}
 Description \\
{\footnotesize
Repeat the step above with false and verify that the laser does not
follow the SMR when it's being moved.\\
LTS-966-REQ-0020.\\[3\baselineskip]

}
\hdashrule[0.5ex]{\textwidth}{1pt}{3mm}
  Expected Result \\
{\footnotesize
The laser should not follow the target

}
\hdashrule[0.5ex]{\textwidth}{1pt}{3mm}
  Actual Result \\
{\footnotesize
There is a misunderstanding of what the SET\_STATION\_LOCK command is
expected to do. This test step was skipped because the expected result
is unrelated to what the step specifies.

}
\begin{tabular}{p{2cm}p{14cm}}
\toprule
Step 8 & Step Execution Status: \textbf{ Not Executed } \\ \hline
\end{tabular}
 Description \\
{\footnotesize
Trigger error 317? LTS-966-REQ-0020\\[2\baselineskip]

}
\hdashrule[0.5ex]{\textwidth}{1pt}{3mm}
  Expected Result \\
{\footnotesize
There is no error as a result of the SET\_STATION\_LOCK command.

}
\hdashrule[0.5ex]{\textwidth}{1pt}{3mm}
  Actual Result \\
{\footnotesize
There is a misunderstanding of what the SET\_STATION\_LOCK command is
expected to do. This test step was skipped because the expected error is
unrelated to what the command actually does.

}
\begin{tabular}{p{2cm}p{14cm}}
\toprule
Step 9 & Step Execution Status: \textbf{ Pass } \\ \hline
\end{tabular}
 Description \\
{\footnotesize
Set the tolerance of the measurement.
SET\_LS\_TOL:\textless{}n;n\textgreater{}. LTS-966-REQ-0021\\
That will allow to define if we need to go in another set of
measurements.\\
1) Give a value greater than 0.1mm to trigger the error\\
2) Give s=0.01mm and verify that this is the right value

}
\hdashrule[0.5ex]{\textwidth}{1pt}{3mm}
  Expected Result \\
{\footnotesize
1) error 311 is triggered\\
2) ACK 300 and the right value is in T2SA

}
\hdashrule[0.5ex]{\textwidth}{1pt}{3mm}
  Actual Result \\
{\footnotesize
Set LS tolerances to rms\_tol = 0.01, max\_tol = 0.02, then measured
M1M3.\\
received

\begin{verbatim}
ERR-311: RMS and Max least squares tolerance value are outside bounds. Tolerances set to defaults.
\end{verbatim}

Reset tolerance to rms\_tol = 0.1, max\_tol = 0.2, then remeasured M1M3.
Received ACK-300.

}
\begin{tabular}{p{2cm}p{14cm}}
\toprule
Step 10 & Step Execution Status: \textbf{ Pass } \\ \hline
\end{tabular}
 Description \\
{\footnotesize
Before measuring the positions, the alignment system publishes the
alt,Az and rot using the command PUBLISH\_ALT\_AZ\_ROT.
LTS-966-REQ-0031\\[2\baselineskip]Repeat the measurement with the
camera.\\
For this step Alt = 0, Az = 0 and rot = 0

}
\hdashrule[0.5ex]{\textwidth}{1pt}{3mm}
  Expected Result \\
{\footnotesize
ACK 300 and verify that T2SA has all 3 values correct.\\
The alignment controller should receive the position of the Camera with
the following format relative to M1M3:\\
\textless{}s\textgreater{};X:\textless{}n\textgreater{};Y:\textless{}n\textgreater{};Z:\textless{}n\textgreater{};Rx:\textless{}n\textgreater{};Ry:\textless{}n\textgreater{};Rz:\textless{}n\textgreater{};\textless{}date\textgreater{}\\[2\baselineskip]

}
\hdashrule[0.5ex]{\textwidth}{1pt}{3mm}
  Actual Result \\
{\footnotesize
Successfully set alt/az/rot and verified all are set to 0 in SA.
Successfully measured camera

}
\begin{tabular}{p{2cm}p{14cm}}
\toprule
Step 11 & Step Execution Status: \textbf{ Pass } \\ \hline
\end{tabular}
 Description \\
{\footnotesize
Measure the position of the M1M3 targets using the command CMD M1M3\\
1) POS\textless{}s, s, s\textgreater{}. LTS-966-REQ-0009\\
2) OFFSET \textless{}s;s;s; s;s;s;\textgreater{}. LTS-966-REQ-0010\\
Repeat for M2 and the camera

}
\hdashrule[0.5ex]{\textwidth}{1pt}{3mm}
  Expected Result \\
{\footnotesize
The alignment controller should receive the position of the M2 with the
following format relative to M1M3:\\
\textless{}s\textgreater{};X:\textbf{\textless{}n\textgreater{}};Y:\textbf{\textless{}n\textgreater{}};Z:\textbf{\textless{}n\textgreater{}};Rx:\textbf{\textless{}n\textgreater{}};Ry:\textbf{\textless{}n\textgreater{}};Rz:\textbf{\textless{}n\textgreater{};\textless{}date\textgreater{}}\\
\textbf{or} the offset OFFSET
(\textless{}s\textgreater{};dX:\textless{}n\textgreater{};dY:\textless{}n\textgreater{};dZ:\textless{}n\textgreater{};dRx:\textless{}n\textgreater{};dRy:\textless{}n\textgreater{};dRz:\textless{}n\textgreater{};\textless{}date\textgreater{}
)\\
Same for the camera\\[2\baselineskip]

}
\hdashrule[0.5ex]{\textwidth}{1pt}{3mm}
  Actual Result \\
{\footnotesize

}
\begin{tabular}{p{2cm}p{14cm}}
\toprule
Step 12 & Step Execution Status: \textbf{ Blocked } \\ \hline
\end{tabular}
 Description \\
{\footnotesize
Block an SMR and repeat the test above.\\
Clear the error before continuing.

}
\hdashrule[0.5ex]{\textwidth}{1pt}{3mm}
  Expected Result \\
{\footnotesize
The measurement should fail and give the err-305.

}
\hdashrule[0.5ex]{\textwidth}{1pt}{3mm}
  Actual Result \\
{\footnotesize
Blocking an SMR did not cause the system to generate the expected error.
Instead, it caused a dialog to pop up on the SA host, which must be
manually clicked. However, this step is considered blocked because this
issue is due to the fact that the system is waiting until it receives
spatial information from at least 3 SMR's. Therefore, Scott will address
this with a coding workaround so that the system recognizes one of the 3
SMR's has been blocked and generates an error as expected.~

}
\hdashrule[0.5ex]{\textwidth}{1pt}{3mm}
  Issues found executing this step:  \\
{\footnotesize
\begin{itemize}
\item \href{https://jira.lsstcorp.org/browse/LVV-19977}{LVV-19977}~~Blocked SMR Coding Workaround

\end{itemize}
}
\begin{tabular}{p{2cm}p{14cm}}
\toprule
Step 13 & Step Execution Status: \textbf{ Pass } \\ \hline
\end{tabular}
 Description \\
{\footnotesize
Send the command PUBLISH\_ALT\_AZ\_ROT with no argument\\
LTS-966-REQ-0031

}
\hdashrule[0.5ex]{\textwidth}{1pt}{3mm}
  Expected Result \\
{\footnotesize
Check that we get the error 320

}
\hdashrule[0.5ex]{\textwidth}{1pt}{3mm}
  Actual Result \\
{\footnotesize
Checked that we get Error 320 when we send blank strings instead of
numeric values, and also when we send things other than numbers.~

}
\begin{tabular}{p{2cm}p{14cm}}
\toprule
Step 14 & Step Execution Status: \textbf{ Pass } \\ \hline
\end{tabular}
 Description \\
{\footnotesize
Generate the report.\\
After finishing up the testing, we want to generate a report by issuing
the command GEN\_REPORT with the report name.\\
LTS-966-REQ-0023

}
\hdashrule[0.5ex]{\textwidth}{1pt}{3mm}
  Expected Result \\
{\footnotesize
The report exists.

}
\hdashrule[0.5ex]{\textwidth}{1pt}{3mm}
  Actual Result \\
{\footnotesize
THe report exists, received~

\begin{verbatim}
ACK-300
\end{verbatim}

}
\begin{tabular}{p{2cm}p{14cm}}
\toprule
Step 15 & Step Execution Status: \textbf{ Blocked } \\ \hline
\end{tabular}
 Description \\
{\footnotesize
Check the error when trying to generate the report by using a template
name that does not exist.\\
Then clear the error\\
LTS-966-REQ-0023

}
\hdashrule[0.5ex]{\textwidth}{1pt}{3mm}
  Expected Result \\
{\footnotesize
Err-308\\
Then clear error and check that the system is READY

}
\hdashrule[0.5ex]{\textwidth}{1pt}{3mm}
  Actual Result \\
{\footnotesize
The report template name isn't supplied by ASC. It's internally stored
by T2SA and we don't interact with it at all. If it's missing, it's a
problem with the T2SA installationn

}
\hdashrule[0.5ex]{\textwidth}{1pt}{3mm}
  Issues found executing this step:  \\
{\footnotesize
\begin{itemize}
\item \href{https://jira.lsstcorp.org/browse/LVV-19978}{LVV-19978}~~Update T2SA Test Steps

\end{itemize}
}
\begin{tabular}{p{2cm}p{14cm}}
\toprule
Step 16 & Step Execution Status: \textbf{ Pass } \\ \hline
\end{tabular}
 Description \\
{\footnotesize
Save the settings by using the command SAVE\_SETTINGS.
LTS-966-REQ-0046\\[2\baselineskip]Error?

}
\hdashrule[0.5ex]{\textwidth}{1pt}{3mm}
  Expected Result \\
{\footnotesize
The settings should be the latest ones.~

}
\hdashrule[0.5ex]{\textwidth}{1pt}{3mm}
  Actual Result \\
{\footnotesize
this forces an immediate writeout of saved settings in T2SA,
guaranteeing changes will be saved even if T2SA crashes. Tested this by
changing LS tolerance, saving settings, force-quitting T2SA, and then
checking for the LS tolerance value after relaunching.

}
\begin{tabular}{p{2cm}p{14cm}}
\toprule
Step 17 & Step Execution Status: \textbf{ Fail } \\ \hline
\end{tabular}
 Description \\
{\footnotesize
Save the job using the SAVE\_SA\_JOBFILE using both\\
1) the default (meaning sending no argument) and\\
2) a valid filename.\\
Verify that the job is saved with proper name\\[2\baselineskip]Send also
a non valid filename and check that we get the error 316\\
LTS-966-REQ-0035

}
\hdashrule[0.5ex]{\textwidth}{1pt}{3mm}
  Expected Result \\
{\footnotesize
The jobs are saved unless the filename is incorrect in which case we
would get the error 316.

}
\hdashrule[0.5ex]{\textwidth}{1pt}{3mm}
  Actual Result \\
{\footnotesize
Strictly speaking, we are not allowed to call the MTAlignment's python
method with no argument. We can implement a default value in the
MTAlignment, and can send an empty string as the argument, which causes
ERR 316. Command is also producing Err316 with what I am pretty sure
should be a valid filename though, so I'm marking this as a fail.

}
\hdashrule[0.5ex]{\textwidth}{1pt}{3mm}
  Issues found executing this step:  \\
{\footnotesize
\begin{itemize}
\item \href{https://jira.lsstcorp.org/browse/LVV-19979}{LVV-19979}~~T2SA SAVE\_SA\_JOBFILE Command

\end{itemize}
}
\begin{tabular}{p{2cm}p{14cm}}
\toprule
Step 18 & Step Execution Status: \textbf{ Not Executed } \\ \hline
\end{tabular}
 Description \\
{\footnotesize
Testing of the command SET\_MEAS\_INDEX to 4. LTS-966-REQ-0043\\
Then repeat the SAVE\_SA\_JOBFILE.

}
\hdashrule[0.5ex]{\textwidth}{1pt}{3mm}
  Expected Result \\
{\footnotesize
The index is set to 4.

}
\hdashrule[0.5ex]{\textwidth}{1pt}{3mm}
  Actual Result \\
{\footnotesize
This step was not able to be executed due to the pandemic. This step was
executed later as part of the continuation of the Factory Acceptance
Test in July 2021 per the~\emph{Testing of the T2SA with the laser
tracker at 45 degrees~}test case.

}
\begin{tabular}{p{2cm}p{14cm}}
\toprule
Step 19 & Step Execution Status: \textbf{ Not Executed } \\ \hline
\end{tabular}
 Description \\
{\footnotesize
Is there an error for SET\_MEAS\_INDEX?\\
LTS-966-REQ-0043

}
\hdashrule[0.5ex]{\textwidth}{1pt}{3mm}
  Expected Result \\
{\footnotesize
The command results in either ACK300 or ERR-333.

}
\hdashrule[0.5ex]{\textwidth}{1pt}{3mm}
  Actual Result \\
{\footnotesize
This step was not able to be executed due to the pandemic. This step was
executed later as part of the continuation of the Factory Acceptance
Test in July 2021 per the \emph{Testing of the T2SA with the laser
tracker at 45 degrees~}test case.

}
\begin{tabular}{p{2cm}p{14cm}}
\toprule
Step 20 & Step Execution Status: \textbf{ Not Executed } \\ \hline
\end{tabular}
 Description \\
{\footnotesize
Testing of the command INC\_MEAS\_INDEX. LTS-966-REQ-0027

}
\hdashrule[0.5ex]{\textwidth}{1pt}{3mm}
  Expected Result \\
{\footnotesize
The increment is set and returns ACK300.

}
\hdashrule[0.5ex]{\textwidth}{1pt}{3mm}
  Actual Result \\
{\footnotesize
This step was not able to be executed due to the pandemic. This step was
executed later as part of the continuation of the Factory Acceptance
Test in July 2021 per the \emph{Testing of the T2SA with the laser
tracker at 45 degrees~}test case.

}
\begin{tabular}{p{2cm}p{14cm}}
\toprule
Step 21 & Step Execution Status: \textbf{ Not Executed } \\ \hline
\end{tabular}
 Description \\
{\footnotesize
Check the command SET\_POWER\_LOCK: 0 or 1\\
LTS-966-REQ-0045

}
\hdashrule[0.5ex]{\textwidth}{1pt}{3mm}
  Expected Result \\
{\footnotesize
The tracker's camera is able to be enabled and disabled, returning
ACK300.

}
\hdashrule[0.5ex]{\textwidth}{1pt}{3mm}
  Actual Result \\
{\footnotesize
This step was not able to be executed due to the pandemic. This step was
executed later as part of the continuation of the Factory Acceptance
Test in July 2021 per the \emph{Testing of the T2SA with the laser
tracker at 45 degrees~}test case.

}

\paragraph{ LVV-T1817 - Motion tests }\mbox{}\\

Version \textbf{1}.
Open  \href{https://jira.lsstcorp.org/secure/Tests.jspa#/testCase/LVV-T1817}{\textit{ LVV-T1817 } }
test case in Jira.

In this test case we will move the targets in different direction and
record the results of the motion\\
- motion in y (vertical)\\
- motion in z\\
- rotation along y\\
- rotation along z\\[2\baselineskip]We can conduct the test with the
laser both in a vertical position and a horizontal position

\textbf{ Preconditions}:\\
The following parameters have to be adjusted:

\begin{itemize}
\tightlist
\item
  SET\_DRIFT\_TOL should be set to a higher value to allow the motion of
  the order that are allowed in the tunnel
\item
  SET\_NUM\_ITERATIONS to 1
\item
  SET\_NUM\_SAMPLES to 3
\item
  Randomization on
\item
  Power-lock off
\end{itemize}

Execution status: {\bf Pass }

Final comment:\\The results of this test execution were captured while the laser tracker
was set up in a vertical position. These steps were repeated when the
laser tracker was reconfigured with a 45degree tilt in the \emph{Testing
of the T2SA with the laser tracker at 45 degrees} test case.


Detailed steps results:

\begin{tabular}{p{2cm}p{14cm}}
\toprule
Step 1 & Step Execution Status: \textbf{ Pass } \\ \hline
\end{tabular}
 Description \\
{\footnotesize
Make a first measurement to get the baseline of the M2 and the Camera
position relative to M1M3

}
\hdashrule[0.5ex]{\textwidth}{1pt}{3mm}
  Expected Result \\
{\footnotesize
The position of the camera in reference to the camera frame is captured.

}
\hdashrule[0.5ex]{\textwidth}{1pt}{3mm}
  Actual Result \\
{\footnotesize
On first attempt, we forgot to set the camera rotation to 0, (it was 5
deg from a previous test). Unexpectedly, the tracker did not attempt to
find the SMRs which are only slightly differently positioned from where
we expect.\\
Initial camera position, in camera frame of reference:\\

\begin{verbatim}
'ACK-300 RefFrame:FrameCAM_0.00_0.00_0.00_2;X:-0.099417;Y:0.009627;Z:-0.112795;Rx:0.006489;Ry:0.053660;Rz:-0.001585;03/12/2020 11:14:10\r\n'
\end{verbatim}

But resolved this by correcting the camera rotation

}
\begin{tabular}{p{2cm}p{14cm}}
\toprule
Step 2 & Step Execution Status: \textbf{ Pass } \\ \hline
\end{tabular}
 Description \\
{\footnotesize
Move the structure in y axis by about 10mm if possible.\\
Then take the same measurement and record the position. If there is a
ruler available, compare the motion measured with the actual motion.\\
** Then try and bring it back to nominal by moving it back by half of
the value that was measured.\\
Take another measurement **\\[2\baselineskip]Repeat ** ** until we're in
within 1mm.\\[2\baselineskip]\textasciitilde{}5 revolutions is about 1
inch

}
\hdashrule[0.5ex]{\textwidth}{1pt}{3mm}
  Expected Result \\
{\footnotesize
The measurements are taken and are within 1mm as expected.

}
\hdashrule[0.5ex]{\textwidth}{1pt}{3mm}
  Actual Result \\
{\footnotesize
We will move the camera structure by +2 revolutions up
(\textasciitilde{}8mm)\\
Initially we ran into trouble because the tracker does not seem to be
searching for the SMR at all\ldots{} Scott worked on a fix\\
Then got results\\

\begin{verbatim}
initial measurement 
ACK-300 RefFrame:FrameCAM_0.00_0.00_0.00_5;X:-0.071180;Y:7.455298;Z:-0.112868;Rx:0.005206;Ry:0.016533;Rz:0.000657;03/12/2020 12:37:13\r\n'
\end{verbatim}

After first iteration moving back toward nominal:\\

\begin{verbatim}
'ACK-300 RefFrame:FrameCAM_0.00_0.00_0.00_6;X:-0.068382;Y:3.901858;Z:-0.116776;Rx:0.006854;Ry:0.038436;Rz:-0.002021;03/12/2020 12:41:03\r\n'
\end{verbatim}

After second iteration\\

\begin{verbatim}
'ACK-300 RefFrame:FrameCAM_0.00_0.00_0.00_7;X:-0.104933;Y:0.712325;Z:-0.123621;Rx:0.006169;Ry:0.046394;Rz:-0.000799;03/12/2020 12:42:13\r\n'
\end{verbatim}

After third iteration\\

\begin{verbatim}
'ACK-300 RefFrame:FrameCAM_0.00_0.00_0.00_8;X:-0.097767;Y:-0.035330;Z:-0.120993;Rx:0.006369;Ry:0.053244;Rz:-0.001400;03/12/2020 12:43:14\r\n'
\end{verbatim}

}
\begin{tabular}{p{2cm}p{14cm}}
\toprule
Step 3 & Step Execution Status: \textbf{ Pass } \\ \hline
\end{tabular}
 Description \\
{\footnotesize
Move the structure in z axis by less than a 10mm if possible.\\
Then take the same measurement and record the position.\\
** Then try and bring it back to nominal by moving it back by half of
the value that was measured.\\
Take another measurement **\\[2\baselineskip]Repeat ** ** until we're in
within 1mm.\\[2\baselineskip]{[}There is a dial indicator allowing to
move the cart by +/- 0.5 inches. Note that we need to keep track of the
number of revolution of the arrow as each revolution is
\textasciitilde{}0.050 inches\\
.6 inches on the way toward us (12 revolution)\\
.4 inches on the way back{]}\\[2\baselineskip]

}
\hdashrule[0.5ex]{\textwidth}{1pt}{3mm}
  Expected Result \\
{\footnotesize
The measurements are taken and are within 1mm as expected.

}
\hdashrule[0.5ex]{\textwidth}{1pt}{3mm}
  Actual Result \\
{\footnotesize
Took initial measuremnts on Cam and M2, dial was at 4.\\

\begin{verbatim}
'ACK-300 RefFrame:FrameM2_0.00_0.00_5.00_9;X:-0.256686;Y:-0.116646;Z:-0.062667;Rx:0.002226;Ry:0.081264;Rz:0.003958;03/12/2020 13:01:07\r\n'
\end{verbatim}

\begin{verbatim}
'ACK-300 RefFrame:FrameCAM_0.00_0.00_5.00_9;X:-0.100149;Y:-0.033987;Z:-0.121444;Rx:0.001629;Ry:0.053691;Rz:-5.001706;03/12/2020 13:01:07\r\n'
\end{verbatim}

then cart into the tunnel until dial was at
42.\\[2\baselineskip]Incremented measurement index, then remeasured cam
and M2\\[2\baselineskip]

\begin{verbatim}
'ACK-300 RefFrame:FrameM2_0.00_0.00_5.00_10;X:-0.198055;Y:-0.001666;Z:7.870649;Rx:0.002616;Ry:0.080003;Rz:0.004325;03/12/2020 13:05:04\r\n'
\end{verbatim}

\begin{verbatim}
'ACK-300 RefFrame:FrameCAM_0.00_0.00_5.00_10;X:-0.172625;Y:-0.487319;Z:7.803832;Rx:0.002056;Ry:0.052748;Rz:-5.001263;03/12/2020 13:05:03\r\n
\end{verbatim}

Tried to move back toward nominal position, but overshot a bit\\

\begin{verbatim}
'ACK-300 RefFrame:FrameCAM_0.00_0.00_5.00_11;X:0.045699;Y:0.881950;Z:-15.735222;Rx:-0.000338;Ry:0.054476;Rz:-5.002011;03/12/2020 13:07:46\r\n'
\end{verbatim}

\begin{verbatim}
'ACK-300 RefFrame:FrameM2_0.00_0.00_5.00_11;X:-0.400159;Y:-0.293055;Z:-15.703755;Rx:0.000292;Ry:0.081652;Rz:0.005893;03/12/2020 13:07:47\r\n'
\end{verbatim}

Final iteration:\\

\begin{verbatim}
'ACK-300 RefFrame:FrameCAM_0.00_0.00_5.00_12;X:-0.010894;Y:0.291359;Z:-5.549460;Rx:0.001370;Ry:0.052778;Rz:-5.002555;03/12/2020 13:09:57\r\n'
\end{verbatim}

\begin{verbatim}
'ACK-300 RefFrame:FrameM2_0.00_0.00_5.00_12;X:-0.300399;Y:-0.178240;Z:-5.501536;Rx:0.001870;Ry:0.079665;Rz:0.004046;03/12/2020 13:09:58\r\n'
\end{verbatim}

}
\begin{tabular}{p{2cm}p{14cm}}
\toprule
Step 4 & Step Execution Status: \textbf{ Not Executed } \\ \hline
\end{tabular}
 Description \\
{\footnotesize
Rotate the structure along the y axis and take a measurement.\\
** Then try and bring it back to nominal by moving it back by half of
the value that was measured.\\
Take another measurement **\\[2\baselineskip]Repeat ** ** until we're in
within 1mm.

}
\hdashrule[0.5ex]{\textwidth}{1pt}{3mm}
  Expected Result \\
{\footnotesize
The measurements are taken and are within 1mm as expected.~

}
\hdashrule[0.5ex]{\textwidth}{1pt}{3mm}
  Actual Result \\
{\footnotesize
This step was not able to be executed due to the pandemic. This step was
executed later as part of the continuation of the Factory Acceptance
Test in July 2021 per the \emph{Testing of the T2SA with the laser
tracker at 45 degrees~}test case.

}
\begin{tabular}{p{2cm}p{14cm}}
\toprule
Step 5 & Step Execution Status: \textbf{ Pass } \\ \hline
\end{tabular}
 Description \\
{\footnotesize
For the rotation along the z axis, we will simulate the rotation by
swapping Target 1 to Target 2, Target 2 to Target 3 and Target 3 to
Target 1 using the software.\\[2\baselineskip]Then the rotation value
sent toT2SA from the ASC would be 120.\\
Measure the position and verify that the targets in SA have indeed moved
by 120.

}
\hdashrule[0.5ex]{\textwidth}{1pt}{3mm}
  Expected Result \\
{\footnotesize
The measurements are taken and the targets have moved by 120 as
expected.

}
\hdashrule[0.5ex]{\textwidth}{1pt}{3mm}
  Actual Result \\
{\footnotesize
We decided not to do the 120 degree rotation, because our triangle is
not a perfect equilateral triangle and so when we get an out of
tolerance error. Instead, we will simulate a 5 degree camrot movement
without actually moving the targets, and then look for the 5 degrees to
show up in the offsets.\\[2\baselineskip]Initial measurement prior to
rotation\\

\begin{verbatim}
 'ACK-300 RefFrame:FrameCAM_0.00_0.00_0.00_8;X:-0.098292;Y:-0.033101;Z:-0.120204;Rx:0.006344;Ry:0.053388;Rz:-0.001932;03/12/2020 12:50:03\r\n'
\end{verbatim}

Then, incremented index and told T2SA that the camera rotator has moved
5 degrees (but it really hasn't, so we expect a 5 degree z rotation
offset in our next measurement\ldots{}\\[2\baselineskip]Then measured
and got results\\

\begin{verbatim}
'ACK-300 RefFrame:FrameCAM_0.00_0.00_5.00_9;X:-0.099674;Y:-0.033590;Z:-0.121127;Rx:0.001604;Ry:0.053812;Rz:-5.001797;03/12/2020 12:54:20\r\n'
\end{verbatim}

-5.002 degree Rz offset shows up as expected\\[2\baselineskip]

}
\begin{tabular}{p{2cm}p{14cm}}
\toprule
Step 6 & Step Execution Status: \textbf{ Not Executed } \\ \hline
\end{tabular}
 Description \\
{\footnotesize
Send the command PUBLISH\_ALT\_AZ\_ROT with ROT = 5deg\\
and repeat the test on the camera

}
\hdashrule[0.5ex]{\textwidth}{1pt}{3mm}
  Expected Result \\
{\footnotesize
The command is accepted and the tracker locates the position of the
SMRs.

}
\hdashrule[0.5ex]{\textwidth}{1pt}{3mm}
  Actual Result \\
{\footnotesize
This step was not able to be executed due to the pandemic. This step was
executed later as part of the continuation of the Factory Acceptance
Test in July 2021 per the \emph{Testing of the T2SA with the laser
tracker at 45 degrees~}test case.

}

\paragraph{ LVV-T1813 - Communication Protocol Interface Command: getting started }\mbox{}\\

Version \textbf{1}.
Open  \href{https://jira.lsstcorp.org/secure/Tests.jspa#/testCase/LVV-T1813}{\textit{ LVV-T1813 } }
test case in Jira.

The objective of this test case will be to verify the initial commands
used to set the system up and calibrate it, using a TCP/IP protocol.~

\textbf{ Preconditions}:\\


Execution status: {\bf Fail }

Final comment:\\The results of this test execution were captured during the continuation
of the tests in July 2021 after the laser tracker was reconfigured with
a 45 degree tilt.

Issues found during the execution of LVV-T1813 test case:

\begin{itemize}
\item \href{https://jira.lsstcorp.org/browse/LVV-19986}{LVV-19986}~~Develop Remote Start of SA

\item \href{https://jira.lsstcorp.org/browse/LVV-19987}{LVV-19987}~~Distinction Between Disconnect and Warming Up States

\item \href{https://jira.lsstcorp.org/browse/LVV-19988}{LVV-19988}~~T2SA CMD Verification

\item \href{https://jira.lsstcorp.org/browse/LVV-19989}{LVV-19989}~~Failed ``LST 0'' Command

\end{itemize}

Detailed steps results:

\begin{tabular}{p{2cm}p{14cm}}
\toprule
Step 1 & Step Execution Status: \textbf{ Pass } \\ \hline
\end{tabular}
 Description \\
{\footnotesize
Verification of the communication protocol: Commands and responses
must:\\

\begin{itemize}
\tightlist
\item
  Contain only ASCII printable characters and whitespace
\item
  Messages always end with carriage return (0x0D `\textbackslash{}r') +
  Line feed (0x0A `\textbackslash{}n')
\item
  If the command is rejected T2SA will send back ERR-300 followed by
  carriage return (0x0D `\textbackslash{}r') + Line feed (0x0A
  `\textbackslash{}n'
\item
  Messages always begin with the character `!' or `?'. `!' to represent
  command and `?' to query for a value.
\end{itemize}

}
\hdashrule[0.5ex]{\textwidth}{1pt}{3mm}
  Expected Result \\
{\footnotesize
The commands and responses follow the proper format:\\
Commands and responses must:\\

\begin{itemize}
\tightlist
\item
  Contain only ASCII printable characters and whitespace
\item
  Messages always end with carriage return (0x0D `\textbackslash{}r') +
  Line feed (0x0A `\textbackslash{}n')
\item
  If the command is rejected T2SA will send back ERR-300 followed by
  carriage return (0x0D `\textbackslash{}r') + Line feed (0x0A
  `\textbackslash{}n'
\item
  Messages always begin with the character `!' or `?'. `!' to represent
  command and `?' to query for a value.
\end{itemize}

}
\hdashrule[0.5ex]{\textwidth}{1pt}{3mm}
  Actual Result \\
{\footnotesize
The commands and responses followed the proper format:\\
Commands and responses includes:\\

\begin{itemize}
\tightlist
\item
  Contained only ASCII printable characters and whitespace
\item
  Messages ended with carriage return (0x0D `\textbackslash{}r') + Line
  feed (0x0A `\textbackslash{}n')
\item
  If the command is rejected T2SA did send back ERR-300 followed by
  carriage return (0x0D `\textbackslash{}r') + Line feed (0x0A
  `\textbackslash{}n'
\item
  Messages always began with the character `!' or `?'. `!' to represent
  command and `?' to query for a value.~
\end{itemize}

}
\begin{tabular}{p{2cm}p{14cm}}
\toprule
Step 2 & Step Execution Status: \textbf{ Blocked } \\ \hline
\end{tabular}
 Description \\
{\footnotesize
Start SA remotely from the Alignment system controller (through T2SA).\\
From the Alignment System Controller and with the laser connected, query
the laser status using the ``LSTA'' command, and verify that is
connected. The laser should be off. (LTS-966-REQ-0008).

}
\hdashrule[0.5ex]{\textwidth}{1pt}{3mm}
  Expected Result \\
{\footnotesize
The application SA should start\\
We should not see the return LNC but LOFF\\
This will also show that the T2SA can receive commands from and responds
to the the Alignment system controller

}
\hdashrule[0.5ex]{\textwidth}{1pt}{3mm}
  Actual Result \\
{\footnotesize
This step is blocked because remote start of SA is still in
development.\\
However, the laser status was still found through Query as part of this
test.~

}
\hdashrule[0.5ex]{\textwidth}{1pt}{3mm}
  Issues found executing this step:  \\
{\footnotesize
\begin{itemize}
\item \href{https://jira.lsstcorp.org/browse/LVV-19986}{LVV-19986}~~Develop Remote Start of SA

\end{itemize}
}
\begin{tabular}{p{2cm}p{14cm}}
\toprule
Step 3 & Step Execution Status: \textbf{ Pass } \\ \hline
\end{tabular}
 Description \\
{\footnotesize
Verify the heart beat of the system (LTS-966-REQ-0002)

}
\hdashrule[0.5ex]{\textwidth}{1pt}{3mm}
  Expected Result \\
{\footnotesize
The heartbeat is incremented by T2SA at least every 0.5 sec through a
range o f0x00000000 - 0xffffffff

}
\hdashrule[0.5ex]{\textwidth}{1pt}{3mm}
  Actual Result \\
{\footnotesize
According to requirement LTS-966-REQ-0002, it does provide status
response while SA\_SDK is completing measurement and analysis process.~

}
\begin{tabular}{p{2cm}p{14cm}}
\toprule
Step 4 & Step Execution Status: \textbf{ Pass } \\ \hline
\end{tabular}
 Description \\
{\footnotesize
Disconnect the laser by using the command ``LST 2'' and check that the
return is telling us that the laser is disconnected.
(REQ-LTS-966-0036).\\[3\baselineskip]

}
\hdashrule[0.5ex]{\textwidth}{1pt}{3mm}
  Expected Result \\
{\footnotesize
Return should be LNC. There should also be an acknowledgement
ACK300.\\[3\baselineskip]

}
\hdashrule[0.5ex]{\textwidth}{1pt}{3mm}
  Actual Result \\
{\footnotesize
Returned LNC and ACK300.~

}
\begin{tabular}{p{2cm}p{14cm}}
\toprule
Step 5 & Step Execution Status: \textbf{ Initial Pass } \\ \hline
\end{tabular}
 Description \\
{\footnotesize
Try to turn the laser on, LST 1.\\[2\baselineskip]Also using the command
`STAT', we should see that the laser is in an error mode
LTS-966-REQ-0011.

}
\hdashrule[0.5ex]{\textwidth}{1pt}{3mm}
  Expected Result \\
{\footnotesize
We should get an error. Err-301 for both commands

}
\hdashrule[0.5ex]{\textwidth}{1pt}{3mm}
  Actual Result \\
{\footnotesize
We got Err-323 : Could not start instrument interface. We could not get
trigger Err-301.

}
\begin{tabular}{p{2cm}p{14cm}}
\toprule
Step 6 & Step Execution Status: \textbf{ Initial Pass } \\ \hline
\end{tabular}
 Description \\
{\footnotesize
Clear error using command CLERCL -\textgreater{} This part was ~descoped
for the 2021 run.\\
Reconnect the laser and turn it on using the command ``LST 1''

}
\hdashrule[0.5ex]{\textwidth}{1pt}{3mm}
  Example Code \\
{\footnotesize
LST 1

}
\hdashrule[0.5ex]{\textwidth}{1pt}{3mm}
  Expected Result \\
{\footnotesize
Return should be LON and it should tell us how long we need to wait
until it warms up (warm with the remaining time in sec).\\
We should also get the ACK300 code

}
\hdashrule[0.5ex]{\textwidth}{1pt}{3mm}
  Actual Result \\
{\footnotesize
Returned LON. Also got the ACK300 code. Does not show the warm up
time.\\[2\baselineskip]Used LST-2 to shut everything down. Then, we
unplugged and plugged back the controller to restart it.\\
The PDU will need to be on.

}
\begin{tabular}{p{2cm}p{14cm}}
\toprule
Step 7 & Step Execution Status: \textbf{ Fail } \\ \hline
\end{tabular}
 Description \\
{\footnotesize
Ask for the status of the laser tracker using the command `LSTA',
several time during the warming process.\\
Once the warm up is ready, then use the command `STAT',
LTS-966-REQ-0011.\\[2\baselineskip]Verify that this is done from the
MTAlignment CSC to verify LTS-966-REQ-0012 and LTS-966-REQ-0001
and\textbf{~}LTS-966-REQ-0004.

}
\hdashrule[0.5ex]{\textwidth}{1pt}{3mm}
  Expected Result \\
{\footnotesize
Return should first be WARM, ``i'' seconds. Then when it is warm, it
should say LON.\\
The laser should be in the state READY as a return of the command STAT.

}
\hdashrule[0.5ex]{\textwidth}{1pt}{3mm}
  Actual Result \\
{\footnotesize
Laser in READY state. We got ACK300. We got LNC while warming
up.\\[2\baselineskip]The capability will be moved to CSC. And Scott will
add a functionality to make sure we are able to get the distinction
between the disconnection and the warming up states

}
\hdashrule[0.5ex]{\textwidth}{1pt}{3mm}
  Issues found executing this step:  \\
{\footnotesize
\begin{itemize}
\item \href{https://jira.lsstcorp.org/browse/LVV-19987}{LVV-19987}~~Distinction Between Disconnect and Warming Up States

\end{itemize}
}
\begin{tabular}{p{2cm}p{14cm}}
\toprule
Step 8 & Step Execution Status: \textbf{ Initial Pass } \\ \hline
\end{tabular}
 Description \\
{\footnotesize
Start the measurement with the single point measurement. Choose a
measurement profile name (s) in the alignment system controller and pass
it to the T2SA using the command SINGLE\_POINT\_MEAS\_PROFILE
\textless{}s\textgreater{} (LTS-966-REQ-0019).\\[2\baselineskip]Then
proceed to the single point measurement using MEAS\_SINGLE\_POINT (which
is a position or an offset measurement?).
LTS-966-REQ-0024\\[2\baselineskip]Before the end of the measurement, use
the `STAT' command to check it's taking a
measurement.\\[2\baselineskip]Ask for the position POINT\_POS
\textless{}refFrame\textgreater{}(LTS-966-REQ-0009)\\
Ask for the offset POINT\_ \textless{}refFrame\textgreater{}
(LTS-966-REQ-0010)\\[3\baselineskip]

}
\hdashrule[0.5ex]{\textwidth}{1pt}{3mm}
  Expected Result \\
{\footnotesize
ACK300\\[2\baselineskip]During the measurement we should get EMP as a
return for the `STAT' command.\\[2\baselineskip]After the measurement,
we get the position POS
(\textless{}s\textgreater{};X:\textbf{\textless{}n\textgreater{}};Y:\textbf{\textless{}n\textgreater{}};Z:\textbf{\textless{}n\textgreater{}};Rx:\textbf{\textless{}n\textgreater{}};Ry:\textbf{\textless{}n\textgreater{}};Rz:\textbf{\textless{}n\textgreater{};\textless{}date\textgreater{}}
)\\
we get the offset OFFSET
(\textless{}s\textgreater{};dX:\textbf{\textless{}n\textgreater{}};dY:\textbf{\textless{}n\textgreater{}};dZ:\textbf{\textless{}n\textgreater{}};dRx:\textbf{\textless{}n\textgreater{}};dRy:\textbf{\textless{}n\textgreater{}};dRz:\textbf{\textless{}n\textgreater{};\textless{}date\textgreater{}}
)

}
\hdashrule[0.5ex]{\textwidth}{1pt}{3mm}
  Actual Result \\
{\footnotesize
As expected, we got ACK-300. However, due to the use of Jupyter
notebooks for this test, we were unable to send commands in parallel. We
were still able to get the single point measurement, but this will need
to be fully verified later when we are not limited to the number of
commands we can send at a time.

}
\begin{tabular}{p{2cm}p{14cm}}
\toprule
Step 9 & Step Execution Status: \textbf{ Pass } \\ \hline
\end{tabular}
 Description \\
{\footnotesize
Repeat the previous test blocking the target and ensure you get an error
306. (LTS-966-REQ-0024)

}
\hdashrule[0.5ex]{\textwidth}{1pt}{3mm}
  Expected Result \\
{\footnotesize
Get an error 306

}
\hdashrule[0.5ex]{\textwidth}{1pt}{3mm}
  Actual Result \\
{\footnotesize
We used wrong name for the point we measured and got Err-306.

}
\begin{tabular}{p{2cm}p{14cm}}
\toprule
Step 10 & Step Execution Status: \textbf{ Pass } \\ \hline
\end{tabular}
 Description \\
{\footnotesize
Send a measurement profile that does not exist and check the status.
(LTS-966-REQ-0019)

}
\hdashrule[0.5ex]{\textwidth}{1pt}{3mm}
  Expected Result \\
{\footnotesize
Get the error 307

}
\hdashrule[0.5ex]{\textwidth}{1pt}{3mm}
  Actual Result \\
{\footnotesize
Sending measurement profiles that do not exist consistently results in
an ERR-307.

}
\begin{tabular}{p{2cm}p{14cm}}
\toprule
Step 11 & Step Execution Status: \textbf{ Pass } \\ \hline
\end{tabular}
 Description \\
{\footnotesize
Load the template file LOAD\_SA\_TEMPLATE\_FILE with a valid template
name. LTS-966-REQ-0022

}
\hdashrule[0.5ex]{\textwidth}{1pt}{3mm}
  Expected Result \\
{\footnotesize
ACK300 and check that the file template is the correct one in SA

}
\hdashrule[0.5ex]{\textwidth}{1pt}{3mm}
  Actual Result \\
{\footnotesize
Got the ACK300.~

}
\begin{tabular}{p{2cm}p{14cm}}
\toprule
Step 12 & Step Execution Status: \textbf{ Pass } \\ \hline
\end{tabular}
 Description \\
{\footnotesize
Change the name of the template to a non valid template and repeat the
loading of the template. LTS-966-REQ-0022

}
\hdashrule[0.5ex]{\textwidth}{1pt}{3mm}
  Expected Result \\
{\footnotesize
We should get an error ERR-312

}
\hdashrule[0.5ex]{\textwidth}{1pt}{3mm}
  Actual Result \\
{\footnotesize
We got Err-312.

}
\begin{tabular}{p{2cm}p{14cm}}
\toprule
Step 13 & Step Execution Status: \textbf{ Pass } \\ \hline
\end{tabular}
 Description \\
{\footnotesize
Set up the randomization to false (SET\_RANDOMIZE\_POINTS 0).
LTS-966-REQ-0014\\
SET\_NUM\_ITERATIONS: The number of iterations is set to 1.
LTS-966-REQ-0013\\
SET\_NUM\_SAMPLES:\textless{}i\textgreater{}. The number of sample is
set to 1 LTS-966-REQ-0017\\[3\baselineskip]Measure the position of M1M3
(CMD M1M3). LTS-966-REQ-0039\\
Then ask POS M1M3 as well as the offsets. (These 2 should be
equal)\\[3\baselineskip]Check that POS * OFFSET = ID

}
\hdashrule[0.5ex]{\textwidth}{1pt}{3mm}
  Expected Result \\
{\footnotesize
We should get an acknowledgment (ACK300) and when the measurement is
done we will get the position with he following format:\\
{POS:
\textless{}s\textgreater{};X:}\textbf{{\textless{}n\textgreater{}}}{;Y:}\textbf{{\textless{}n\textgreater{}}}{;Z:}\textbf{{\textless{}n\textgreater{}}}{;Rx:}\textbf{{\textless{}n\textgreater{}}}{;Ry:}\textbf{{\textless{}n\textgreater{}}}{;Rz:}\textbf{{\textless{}n\textgreater{};\textless{}date\textgreater{}}}{~}\\
{OFFSET:~{\textless{}s\textgreater{};dX:}\textbf{{\textless{}n\textgreater{}}}{;dY:}\textbf{{\textless{}n\textgreater{}}}{;dZ:}\textbf{{\textless{}n\textgreater{}}}{;dRx:}\textbf{{\textless{}n\textgreater{}}}{;dRy:}\textbf{{\textless{}n\textgreater{}}}{;dRz:}\textbf{{\textless{}n\textgreater{};\textless{}date\textgreater{}}}{~}}

}
\hdashrule[0.5ex]{\textwidth}{1pt}{3mm}
  Actual Result \\
{\footnotesize
Got ACK300.\\
Measurement done.\\
Position and offset showed.~

}
\begin{tabular}{p{2cm}p{14cm}}
\toprule
Step 14 & Step Execution Status: \textbf{ Fail } \\ \hline
\end{tabular}
 Description \\
{\footnotesize
Use CMD with an unknown name and check the error received.
LTS-966-REQ-0039

}
\hdashrule[0.5ex]{\textwidth}{1pt}{3mm}
  Expected Result \\
{\footnotesize
The error should be 305 or 300?\\[2\baselineskip]

}
\hdashrule[0.5ex]{\textwidth}{1pt}{3mm}
  Actual Result \\
{\footnotesize
T2SA received the command. But, we did not get any error returned.~

}
\hdashrule[0.5ex]{\textwidth}{1pt}{3mm}
  Issues found executing this step:  \\
{\footnotesize
\begin{itemize}
\item \href{https://jira.lsstcorp.org/browse/LVV-19988}{LVV-19988}~~T2SA CMD Verification

\end{itemize}
}
\begin{tabular}{p{2cm}p{14cm}}
\toprule
Step 15 & Step Execution Status: \textbf{ Pass } \\ \hline
\end{tabular}
 Description \\
{\footnotesize
Change the number of the iterations using SET\_NUM\_ITERATIONS 5 for
instance and repeat the M1M3 measurement.
LTS-966-REQ-0013\\[2\baselineskip]

}
\hdashrule[0.5ex]{\textwidth}{1pt}{3mm}
  Expected Result \\
{\footnotesize
ACK300 and ensure that each points were measured 5 times. At the end we
should get the absolute position of M1M3.~

}
\hdashrule[0.5ex]{\textwidth}{1pt}{3mm}
  Actual Result \\
{\footnotesize
We got ACK300 . Each points were measured 5 times. Got the absolute
position of M1M3.

}
\begin{tabular}{p{2cm}p{14cm}}
\toprule
Step 16 & Step Execution Status: \textbf{ Pass } \\ \hline
\end{tabular}
 Description \\
{\footnotesize
Check the error 322. One option is to send a negative value or not an
integer to SET\_NUM\_ITERATIONS. LTS-966-REQ-0013

}
\hdashrule[0.5ex]{\textwidth}{1pt}{3mm}
  Expected Result \\
{\footnotesize
error 322 is triggered

}
\hdashrule[0.5ex]{\textwidth}{1pt}{3mm}
  Actual Result \\
{\footnotesize
We got Err-322.

}
\begin{tabular}{p{2cm}p{14cm}}
\toprule
Step 17 & Step Execution Status: \textbf{ Pass } \\ \hline
\end{tabular}
 Description \\
{\footnotesize
Change the number of samples to 3 using SET\_NUM\_SAMPLES 3 leaving the
iterations are set to 5.\\
LTS-966-REQ-0017

}
\hdashrule[0.5ex]{\textwidth}{1pt}{3mm}
  Expected Result \\
{\footnotesize
ACK300 and ensure that the points are measured 15 times.~

}
\hdashrule[0.5ex]{\textwidth}{1pt}{3mm}
  Actual Result \\
{\footnotesize
We got ACK300. The points were measured 9 times as expected.~

}
\begin{tabular}{p{2cm}p{14cm}}
\toprule
Step 18 & Step Execution Status: \textbf{ Pass } \\ \hline
\end{tabular}
 Description \\
{\footnotesize
Check the error 321. One option is to send a negative value or not an
integer to SET\_NUM\_SAMPLES. LTS-966-REQ-0017

}
\hdashrule[0.5ex]{\textwidth}{1pt}{3mm}
  Expected Result \\
{\footnotesize
error 321 is triggered

}
\hdashrule[0.5ex]{\textwidth}{1pt}{3mm}
  Actual Result \\
{\footnotesize
We got Err-321.

}
\begin{tabular}{p{2cm}p{14cm}}
\toprule
Step 19 & Step Execution Status: \textbf{ Pass } \\ \hline
\end{tabular}
 Description \\
{\footnotesize
Same exercise for the randomization. SET\_RANDOMIZE\_POINTS to TRUE ~and
repeat the measurement of M1M3. LTS-966-REQ-0014\\[2\baselineskip]

}
\hdashrule[0.5ex]{\textwidth}{1pt}{3mm}
  Expected Result \\
{\footnotesize
ACK300 and verify that the order of the measurements is random and
different from the previous step.~

}
\hdashrule[0.5ex]{\textwidth}{1pt}{3mm}
  Actual Result \\
{\footnotesize
We got ACK300. ~Measurements were random from the previous step.

}
\begin{tabular}{p{2cm}p{14cm}}
\toprule
Step 20 & Step Execution Status: \textbf{ Pass } \\ \hline
\end{tabular}
 Description \\
{\footnotesize
Check error 302 by giving the value 0 to it. LTS-966-REQ-0014

}
\hdashrule[0.5ex]{\textwidth}{1pt}{3mm}
  Expected Result \\
{\footnotesize
trigger check error 302

}
\hdashrule[0.5ex]{\textwidth}{1pt}{3mm}
  Actual Result \\
{\footnotesize
We got Err-302 for string value.~

}
\begin{tabular}{p{2cm}p{14cm}}
\toprule
Step 21 & Step Execution Status: \textbf{ Not Executed } \\ \hline
\end{tabular}
 Description \\
{\footnotesize
Repeat the measurement of M1M3 and halt the measurement using HALT.
\textbf{:~}LTS-966-REQ-0016\\
Then check status using the command `STAT'

}
\hdashrule[0.5ex]{\textwidth}{1pt}{3mm}
  Expected Result \\
{\footnotesize
ACK300 and the measurement should stop and the status STAT should be
READY.\\
Err 330????

}
\hdashrule[0.5ex]{\textwidth}{1pt}{3mm}
  Actual Result \\
{\footnotesize
Because we used a Jupyter notebook for this test, we were only able to
send commands in series. Therefore, we were unable to send a HALT
command before the measurement was finished.~

}
\begin{tabular}{p{2cm}p{14cm}}
\toprule
Step 22 & Step Execution Status: \textbf{ Not Executed } \\ \hline
\end{tabular}
 Description \\
{\footnotesize
Trigger of error 319:\\
Use the command `HALT' when the tracker is not measuring.
LTS-966-REQ-0016

}
\hdashrule[0.5ex]{\textwidth}{1pt}{3mm}
  Expected Result \\
{\footnotesize
Error 319

}
\hdashrule[0.5ex]{\textwidth}{1pt}{3mm}
  Actual Result \\
{\footnotesize
Because we used a Jupyter notebook for this test, we were only able to
send commands in series. Therefore, we were unable to provoke ERR-319
using the HALT command.~

}
\begin{tabular}{p{2cm}p{14cm}}
\toprule
Step 23 & Step Execution Status: \textbf{ Pass } \\ \hline
\end{tabular}
 Description \\
{\footnotesize
Proceed to a reset using the command RESET\_T2SA. LTS-966-REQ-0032\\
Then ask the status STAT.

}
\hdashrule[0.5ex]{\textwidth}{1pt}{3mm}
  Expected Result \\
{\footnotesize
ACK 300 and the system is reset. At the end of the process the status of
the system should be READY.

}
\hdashrule[0.5ex]{\textwidth}{1pt}{3mm}
  Actual Result \\
{\footnotesize
We got ACK 300. We got the status ``READY'' at the end of the process.~

}
\begin{tabular}{p{2cm}p{14cm}}
\toprule
Step 24 & Step Execution Status: \textbf{ Not Executed } \\ \hline
\end{tabular}
 Description \\
{\footnotesize
Trigger 318: Proceed to a reset while T2SA is measuring?\\
LTS-966-REQ-0032

}
\hdashrule[0.5ex]{\textwidth}{1pt}{3mm}
  Expected Result \\
{\footnotesize
The T2SA does not trigger ERR-318

}
\hdashrule[0.5ex]{\textwidth}{1pt}{3mm}
  Actual Result \\
{\footnotesize
Because we used a Jupyter notebook for this test, we were only able to
send commands in series. Therefore, we were unable to provoke ERR-318.~

}
\begin{tabular}{p{2cm}p{14cm}}
\toprule
Step 25 & Step Execution Status: \textbf{ Not Executed } \\ \hline
\end{tabular}
 Description \\
{\footnotesize
Move the laser and send the command NEW\_STATION. Verify in SA that the
new station was added and repeat a measurement of M1M3.
LTS-966-REQ-0033\\
Also test the error 329 by covering reflectors?

}
\hdashrule[0.5ex]{\textwidth}{1pt}{3mm}
  Expected Result \\
{\footnotesize
The new station should appear in SA and the measurement of M1M3 should
give the new position

}
\hdashrule[0.5ex]{\textwidth}{1pt}{3mm}
  Actual Result \\
{\footnotesize
Due to the lack of personnel support to safely re-configure the testing
set up, this step was skipped.~

}
\begin{tabular}{p{2cm}p{14cm}}
\toprule
Step 26 & Step Execution Status: \textbf{ Fail } \\ \hline
\end{tabular}
 Description \\
{\footnotesize
Using the command ``LST 0'' Turn the laser off\\
Use LSTA to check that the laser is off

}
\hdashrule[0.5ex]{\textwidth}{1pt}{3mm}
  Expected Result \\
{\footnotesize
ACK 300 and LOFF

}
\hdashrule[0.5ex]{\textwidth}{1pt}{3mm}
  Actual Result \\
{\footnotesize
We got ACK300. Reported the status as LNC instead of LOFF.~

}
\hdashrule[0.5ex]{\textwidth}{1pt}{3mm}
  Issues found executing this step:  \\
{\footnotesize
\begin{itemize}
\item \href{https://jira.lsstcorp.org/browse/LVV-19989}{LVV-19989}~~Failed ``LST 0'' Command

\end{itemize}
}

\paragraph{ LVV-T1815 - Check Procedures Testing }\mbox{}\\

Version \textbf{1}.
Open  \href{https://jira.lsstcorp.org/secure/Tests.jspa#/testCase/LVV-T1815}{\textit{ LVV-T1815 } }
test case in Jira.

The objective of this test case will be to verify the following
commands:

\begin{itemize}
\tightlist
\item
  2FACE
\item
  ADM
\item
  DRIFT
\end{itemize}

\textbf{ Preconditions}:\\


Execution status: {\bf Pass }

Final comment:\\The results of this test execution were captured during the continuation
of the tests in July 2021 after the laser tracker was reconfigured with
a 45 degree tilt.


Detailed steps results:

\begin{tabular}{p{2cm}p{14cm}}
\toprule
Step 1 & Step Execution Status: \textbf{ Pass } \\ \hline
\end{tabular}
 Description \\
{\footnotesize
Execute a 2 Face check using the following command: 2FACE\_CHECK with
the point group. LTS-966-REQ-0028\\
Check the status using STAT

}
\hdashrule[0.5ex]{\textwidth}{1pt}{3mm}
  Expected Result \\
{\footnotesize
There should be an acknowledgment if the check passed the tolerance ACK
300. ~\\
The status while doing the check should be 2Face check

}
\hdashrule[0.5ex]{\textwidth}{1pt}{3mm}
  Actual Result \\
{\footnotesize
We got ACK300 and the status 2 Face Check.

}
\begin{tabular}{p{2cm}p{14cm}}
\toprule
Step 2 & Step Execution Status: \textbf{ Pass } \\ \hline
\end{tabular}
 Description \\
{\footnotesize
Change the tolerance setting for the 2 face check using SET\_2FACE\_TOL.
LTS-966-REQ-0030\\
First set it up to a value outside the bounds (??) and check that we get
an error.\\
Then set it up to a tolerance that will trigger an out of tolerance
value\\
Repeat the 2Face check with this tolerance configuration

}
\hdashrule[0.5ex]{\textwidth}{1pt}{3mm}
  Expected Result \\
{\footnotesize
When the value set is out of tolerance, we should get the error 309\\
Then when we set the value inside the bound we should get ACK300. The
return from the test should be 303.

}
\hdashrule[0.5ex]{\textwidth}{1pt}{3mm}
  Actual Result \\
{\footnotesize
For Tolerance, we got Err-303. Need to change the test description.~

}
\begin{tabular}{p{2cm}p{14cm}}
\toprule
Step 3 & Step Execution Status: \textbf{ Pass } \\ \hline
\end{tabular}
 Description \\
{\footnotesize
SET\_DRIFT\_TOL to 0.1 and check that the value that is in T2SA is 0.1.
~LTS-966-REQ-0034\\
Then measure the drift using the MEAS\_DRIFT command. LTS-966-REQ-0026

}
\hdashrule[0.5ex]{\textwidth}{1pt}{3mm}
  Expected Result \\
{\footnotesize
The drift test should start and if the drift test passes the return
should be 300.~

}
\hdashrule[0.5ex]{\textwidth}{1pt}{3mm}
  Actual Result \\
{\footnotesize
Drift test passed. We got ACK300.

}
\begin{tabular}{p{2cm}p{14cm}}
\toprule
Step 4 & Step Execution Status: \textbf{ Pass } \\ \hline
\end{tabular}
 Description \\
{\footnotesize
Set the tolerance to a small value (0.01?) and rerun the test. The value
should be small enough to make the drift test fail.~

}
\hdashrule[0.5ex]{\textwidth}{1pt}{3mm}
  Expected Result \\
{\footnotesize
error 305 or 310 should be triggered

}
\hdashrule[0.5ex]{\textwidth}{1pt}{3mm}
  Actual Result \\
{\footnotesize
We got Err-304 as expected.~

}
\begin{tabular}{p{2cm}p{14cm}}
\toprule
Step 5 & Step Execution Status: \textbf{ Not Executed } \\ \hline
\end{tabular}
 Description \\
{\footnotesize
Run through the ADM (Command?)\\
Check the status during the test: STAT

}
\hdashrule[0.5ex]{\textwidth}{1pt}{3mm}
  Expected Result \\
{\footnotesize
Result from Status: ADM

}
\hdashrule[0.5ex]{\textwidth}{1pt}{3mm}
  Actual Result \\
{\footnotesize
This step was not executed because the ADM as a command has since been
descoped

}

\paragraph{ LVV-T2181 - Testing of the T2SA with the laser tracker at 45degrees }\mbox{}\\

Version \textbf{1}.
Open  \href{https://jira.lsstcorp.org/secure/Tests.jspa#/testCase/LVV-T2181}{\textit{ LVV-T2181 } }
test case in Jira.

The objective of this test case is to re-verify some of the steps that
were initially done as part of the FAT testing in March 2020 but with
the laser tracker tilted at 45deg.~

\textbf{ Preconditions}:\\


Execution status: {\bf Blocked }

Final comment:\\As mentioned by the objective, this test case is meant to redo some of
the original testing that was done when the laser tracker was set up
horizontally. Specifically, the~\emph{Position Measurement of M1M3, M2
and the camera~}test case and the \emph{Motion tests} test case have
been called to test and will be repeated with the laser tracker tilted
at 45deg.

Issues found during the execution of LVV-T2181 test case:

\begin{itemize}
\item \href{https://jira.lsstcorp.org/browse/LVV-19978}{LVV-19978}~~Update T2SA Test Steps

\end{itemize}

Detailed steps results:

\begin{tabular}{p{2cm}p{14cm}}
\toprule
Step 1 & Step Execution Status: \textbf{ Pass } \\ \hline
\end{tabular}
 Description \\
{\footnotesize
To start a measurement, one needs to set the reference group of point
using the command SET\_REFERENCE\_GROUP. LTS-966-REQ-0015\\
Start without specifying a group name to trigger the
error\\[2\baselineskip]

}
\hdashrule[0.5ex]{\textwidth}{1pt}{3mm}
  Expected Result \\
{\footnotesize
Error 313 should be triggered

}
\hdashrule[0.5ex]{\textwidth}{1pt}{3mm}
  Actual Result \\
{\footnotesize
Expected error should be Err-306( which we got).~

}
\begin{tabular}{p{2cm}p{14cm}}
\toprule
Step 2 & Step Execution Status: \textbf{ Pass } \\ \hline
\end{tabular}
 Description \\
{\footnotesize
Repeat the step above with a valid group name. LTS-966-REQ-0015

}
\hdashrule[0.5ex]{\textwidth}{1pt}{3mm}
  Expected Result \\
{\footnotesize
​​​​Verify in SA that the group name is the correct one

}
\hdashrule[0.5ex]{\textwidth}{1pt}{3mm}
  Actual Result \\
{\footnotesize
We got ACK300.~

}
\begin{tabular}{p{2cm}p{14cm}}
\toprule
Step 3 & Step Execution Status: \textbf{ Pass } \\ \hline
\end{tabular}
 Description \\
{\footnotesize
Next, one needs to set the working frame using SET\_WORKING\_FRAME.
Start without specifying a group name to trigger the error
LTS-966-REQ-0018

}
\hdashrule[0.5ex]{\textwidth}{1pt}{3mm}
  Expected Result \\
{\footnotesize
Error 314 should be triggered

}
\hdashrule[0.5ex]{\textwidth}{1pt}{3mm}
  Actual Result \\
{\footnotesize
We got Err-314.

}
\begin{tabular}{p{2cm}p{14cm}}
\toprule
Step 4 & Step Execution Status: \textbf{ Pass } \\ \hline
\end{tabular}
 Description \\
{\footnotesize
Repeat the test from above giving a working frame with the right format~

}
\hdashrule[0.5ex]{\textwidth}{1pt}{3mm}
  Expected Result \\
{\footnotesize
Verify in SA that the working frame is the correct one. ACK 300

}
\hdashrule[0.5ex]{\textwidth}{1pt}{3mm}
  Actual Result \\
{\footnotesize
We got ACK300.

}
\begin{tabular}{p{2cm}p{14cm}}
\toprule
Step 5 & Step Execution Status: \textbf{ Not Executed } \\ \hline
\end{tabular}
 Description \\
{\footnotesize
Use the command SET\_STATION\_LOCK with true and check that the laser is
still locked on a SMR even when we move them LTS-966-REQ-0020

}
\hdashrule[0.5ex]{\textwidth}{1pt}{3mm}
  Expected Result \\
{\footnotesize
The laser should follow the target

}
\hdashrule[0.5ex]{\textwidth}{1pt}{3mm}
  Actual Result \\
{\footnotesize
There is a misunderstanding of what the SET\_STATION\_LOCK command is
expected to do. This test step was skipped because the expected result
is unrelated to what the step specifies.

}
\begin{tabular}{p{2cm}p{14cm}}
\toprule
Step 6 & Step Execution Status: \textbf{ Not Executed } \\ \hline
\end{tabular}
 Description \\
{\footnotesize
Check of error 331, fail to lock station ?

}
\hdashrule[0.5ex]{\textwidth}{1pt}{3mm}
  Expected Result \\
{\footnotesize
There is no error as a result of the SET\_STATION\_LOCK command.~

}
\hdashrule[0.5ex]{\textwidth}{1pt}{3mm}
  Actual Result \\
{\footnotesize
There is a misunderstanding of what the SET\_STATION\_LOCK command is
expected to do. This test step was skipped because the expected result
is unrelated to what the step specifies.

}
\begin{tabular}{p{2cm}p{14cm}}
\toprule
Step 7 & Step Execution Status: \textbf{ Not Executed } \\ \hline
\end{tabular}
 Description \\
{\footnotesize
Repeat the step above with false and verify that the laser does not
follow the SMR when it's being moved.\\
LTS-966-REQ-0020.\\[3\baselineskip]

}
\hdashrule[0.5ex]{\textwidth}{1pt}{3mm}
  Expected Result \\
{\footnotesize
The laser should not follow the target

}
\hdashrule[0.5ex]{\textwidth}{1pt}{3mm}
  Actual Result \\
{\footnotesize
There is a misunderstanding of what the SET\_STATION\_LOCK command is
expected to do. This test step was skipped because the expected result
is unrelated to what the step specifies.

}
\begin{tabular}{p{2cm}p{14cm}}
\toprule
Step 8 & Step Execution Status: \textbf{ Not Executed } \\ \hline
\end{tabular}
 Description \\
{\footnotesize
Trigger error 317? LTS-966-REQ-0020\\[2\baselineskip]

}
\hdashrule[0.5ex]{\textwidth}{1pt}{3mm}
  Expected Result \\
{\footnotesize
There is no error as a result of the SET\_STATION\_LOCK command.

}
\hdashrule[0.5ex]{\textwidth}{1pt}{3mm}
  Actual Result \\
{\footnotesize
There is a misunderstanding of what the SET\_STATION\_LOCK command is
expected to do. This test step was skipped because the expected result
is unrelated to what the step specifies.

}
\begin{tabular}{p{2cm}p{14cm}}
\toprule
Step 9 & Step Execution Status: \textbf{ Pass } \\ \hline
\end{tabular}
 Description \\
{\footnotesize
Set the tolerance of the measurement.
SET\_LS\_TOL:\textless{}n;n\textgreater{}. LTS-966-REQ-0021\\
That will allow to define if we need to go in another set of
measurements.\\
1) Give a value greater than 0.1mm to trigger the error\\
2) Give s=0.01mm and verify that this is the right value

}
\hdashrule[0.5ex]{\textwidth}{1pt}{3mm}
  Expected Result \\
{\footnotesize
1) error 311 is triggered\\
2) ACK 300 and the right value is in T2SA

}
\hdashrule[0.5ex]{\textwidth}{1pt}{3mm}
  Actual Result \\
{\footnotesize
rms\_tol = 0.01, max\_tol = 0.02\\
We got Error 311 in response to asking for measurement, but not for
setting tolerance.~\\
We got ACK300.\\[2\baselineskip]

}
\begin{tabular}{p{2cm}p{14cm}}
\toprule
Step 10 & Step Execution Status: \textbf{ Pass } \\ \hline
\end{tabular}
 Description \\
{\footnotesize
Before measuring the positions, the alignment system publishes the
alt,Az and rot using the command PUBLISH\_ALT\_AZ\_ROT.
LTS-966-REQ-0031\\[2\baselineskip]Repeat the measurement with the
camera.\\
For this step Alt = 0, Az = 0 and rot = 0

}
\hdashrule[0.5ex]{\textwidth}{1pt}{3mm}
  Expected Result \\
{\footnotesize
ACK 300 and verify that T2SA has all 3 values correct.\\
The alignment controller should receive the position of the Camera with
the following format relative to M1M3:\\
\textless{}s\textgreater{};X:\textless{}n\textgreater{};Y:\textless{}n\textgreater{};Z:\textless{}n\textgreater{};Rx:\textless{}n\textgreater{};Ry:\textless{}n\textgreater{};Rz:\textless{}n\textgreater{};\textless{}date\textgreater{}\\[2\baselineskip]

}
\hdashrule[0.5ex]{\textwidth}{1pt}{3mm}
  Actual Result \\
{\footnotesize
We got ACK300 and Position of the camera with proper format.

}
\begin{tabular}{p{2cm}p{14cm}}
\toprule
Step 11 & Step Execution Status: \textbf{ Pass } \\ \hline
\end{tabular}
 Description \\
{\footnotesize
Measure the position of the M1M3 targets using the command CMD M1M3\\
1) POS\textless{}s, s, s\textgreater{}. LTS-966-REQ-0009\\
2) OFFSET \textless{}s;s;s; s;s;s;\textgreater{}. LTS-966-REQ-0010\\
Repeat for M2 and the camera

}
\hdashrule[0.5ex]{\textwidth}{1pt}{3mm}
  Expected Result \\
{\footnotesize
The alignment controller should receive the position of the M2 with the
following format relative to M1M3:\\
\textless{}s\textgreater{};X:\textbf{\textless{}n\textgreater{}};Y:\textbf{\textless{}n\textgreater{}};Z:\textbf{\textless{}n\textgreater{}};Rx:\textbf{\textless{}n\textgreater{}};Ry:\textbf{\textless{}n\textgreater{}};Rz:\textbf{\textless{}n\textgreater{};\textless{}date\textgreater{}}\\
\textbf{or} the offset OFFSET
(\textless{}s\textgreater{};dX:\textless{}n\textgreater{};dY:\textless{}n\textgreater{};dZ:\textless{}n\textgreater{};dRx:\textless{}n\textgreater{};dRy:\textless{}n\textgreater{};dRz:\textless{}n\textgreater{};\textless{}date\textgreater{}
)\\
Same for the camera\\[2\baselineskip]

}
\hdashrule[0.5ex]{\textwidth}{1pt}{3mm}
  Actual Result \\
{\footnotesize
We got ACK300. We got position and offset values for all three targets.

}
\begin{tabular}{p{2cm}p{14cm}}
\toprule
Step 12 & Step Execution Status: \textbf{ Initial Pass } \\ \hline
\end{tabular}
 Description \\
{\footnotesize
Block an SMR and repeat the test above.\\
Clear the error before continuing.

}
\hdashrule[0.5ex]{\textwidth}{1pt}{3mm}
  Expected Result \\
{\footnotesize
The measurement should fail and give the err-305.

}
\hdashrule[0.5ex]{\textwidth}{1pt}{3mm}
  Actual Result \\
{\footnotesize
It was initially seen from the \emph{Position Measurement of M1M3, M2
and the camera} test done in March 2020 that using three total SMR's for
this test impacted our ability to verify the error code. However, the
vendor introduced a workaround in the code to allow for us to verify
that an error could be provoked using only three SMR's. As a result, we
received ERR-326 ``Fail Least Squares Best Fit''. Although this was not
the expected result, this step was still seen as an initial pass because
this is the expected error when the laser tracker can only detect 2
total SMR's. The laser tracker expects to detect a total of 12 total
SMR's so ERR-305 would only be given when there are 11 SMR's detected

}
\begin{tabular}{p{2cm}p{14cm}}
\toprule
Step 13 & Step Execution Status: \textbf{ Pass } \\ \hline
\end{tabular}
 Description \\
{\footnotesize
Send the command PUBLISH\_ALT\_AZ\_ROT with no argument\\
LTS-966-REQ-0031

}
\hdashrule[0.5ex]{\textwidth}{1pt}{3mm}
  Expected Result \\
{\footnotesize
Check that we get the error 320

}
\hdashrule[0.5ex]{\textwidth}{1pt}{3mm}
  Actual Result \\
{\footnotesize
We got Err-320 as expected.~\\[2\baselineskip]

}
\begin{tabular}{p{2cm}p{14cm}}
\toprule
Step 14 & Step Execution Status: \textbf{ Pass } \\ \hline
\end{tabular}
 Description \\
{\footnotesize
Generate the report.\\
After finishing up the testing, we want to generate a report by issuing
the command GEN\_REPORT with the report name.\\
LTS-966-REQ-0023

}
\hdashrule[0.5ex]{\textwidth}{1pt}{3mm}
  Expected Result \\
{\footnotesize
The report exists.

}
\hdashrule[0.5ex]{\textwidth}{1pt}{3mm}
  Actual Result \\
{\footnotesize
the report exists ( in spatial analyzer under the Reports collection)

}
\begin{tabular}{p{2cm}p{14cm}}
\toprule
Step 15 & Step Execution Status: \textbf{ Blocked } \\ \hline
\end{tabular}
 Description \\
{\footnotesize
Check the error when trying to generate the report by using a template
name that does not exist.\\
Then clear the error\\
LTS-966-REQ-0023

}
\hdashrule[0.5ex]{\textwidth}{1pt}{3mm}
  Expected Result \\
{\footnotesize
Err-308\\
Then clear error and check that the system is READY

}
\hdashrule[0.5ex]{\textwidth}{1pt}{3mm}
  Actual Result \\
{\footnotesize
At the time of the test, it wasn't clear how the GEN\_REPORT command was
supposed to be used. It was assumed that the test was to generate a
report without initially specifying a name and so this step was skipped
because the T2SA automatically generates a name for any report missing
one. This step will need to be revised so that it is clear we are
testing that trying to specify an incorrect report name will result in
the expected error.~

}
\hdashrule[0.5ex]{\textwidth}{1pt}{3mm}
  Issues found executing this step:  \\
{\footnotesize
\begin{itemize}
\item \href{https://jira.lsstcorp.org/browse/LVV-19978}{LVV-19978}~~Update T2SA Test Steps

\end{itemize}
}
\begin{tabular}{p{2cm}p{14cm}}
\toprule
Step 16 & Step Execution Status: \textbf{ Pass } \\ \hline
\end{tabular}
 Description \\
{\footnotesize
Save the settings by using the command SAVE\_SETTINGS.
LTS-966-REQ-0046\\[2\baselineskip]Error?

}
\hdashrule[0.5ex]{\textwidth}{1pt}{3mm}
  Expected Result \\
{\footnotesize
The settings should be the latest ones.~

}
\hdashrule[0.5ex]{\textwidth}{1pt}{3mm}
  Actual Result \\
{\footnotesize
Saved settings.

}
\begin{tabular}{p{2cm}p{14cm}}
\toprule
Step 17 & Step Execution Status: \textbf{ Pass } \\ \hline
\end{tabular}
 Description \\
{\footnotesize
Save the job using the SAVE\_SA\_JOBFILE using both\\
1) the default (meaning sending no argument) and\\
2) a valid filename.\\
Verify that the job is saved with proper name\\[2\baselineskip]Send also
a non valid filename and check that we get the error 316\\
LTS-966-REQ-0035

}
\hdashrule[0.5ex]{\textwidth}{1pt}{3mm}
  Expected Result \\
{\footnotesize
The jobs are saved unless the filename is incorrect in which case we
would get the error 316.

}
\hdashrule[0.5ex]{\textwidth}{1pt}{3mm}
  Actual Result \\
{\footnotesize
We got Err-316. Got ACK300 for valine filename.~

}
\begin{tabular}{p{2cm}p{14cm}}
\toprule
Step 18 & Step Execution Status: \textbf{ Pass } \\ \hline
\end{tabular}
 Description \\
{\footnotesize
Testing of the command SET\_MEAS\_INDEX to 4. LTS-966-REQ-0043\\
Then repeat the SAVE\_SA\_JOBFILE.

}
\hdashrule[0.5ex]{\textwidth}{1pt}{3mm}
  Expected Result \\
{\footnotesize
The index is set to 4.

}
\hdashrule[0.5ex]{\textwidth}{1pt}{3mm}
  Actual Result \\
{\footnotesize
We got ACK300. Index set to 4.\\
Saved the job file.~\\[2\baselineskip]

}
\begin{tabular}{p{2cm}p{14cm}}
\toprule
Step 19 & Step Execution Status: \textbf{ Pass } \\ \hline
\end{tabular}
 Description \\
{\footnotesize
Is there an error for SET\_MEAS\_INDEX?\\
LTS-966-REQ-0043

}
\hdashrule[0.5ex]{\textwidth}{1pt}{3mm}
  Expected Result \\
{\footnotesize
The command results in either ACK300 or ERR-333.

}
\hdashrule[0.5ex]{\textwidth}{1pt}{3mm}
  Actual Result \\
{\footnotesize
Got Error 333 for string . But. it accepts negative number.~

}
\begin{tabular}{p{2cm}p{14cm}}
\toprule
Step 20 & Step Execution Status: \textbf{ Pass } \\ \hline
\end{tabular}
 Description \\
{\footnotesize
Testing of the command INC\_MEAS\_INDEX. LTS-966-REQ-0027

}
\hdashrule[0.5ex]{\textwidth}{1pt}{3mm}
  Expected Result \\
{\footnotesize
The increment is set and returns ACK300.

}
\hdashrule[0.5ex]{\textwidth}{1pt}{3mm}
  Actual Result \\
{\footnotesize
ACK300. Increment worked.~

}
\begin{tabular}{p{2cm}p{14cm}}
\toprule
Step 21 & Step Execution Status: \textbf{ Pass } \\ \hline
\end{tabular}
 Description \\
{\footnotesize
Check the command SET\_POWER\_LOCK: 0 or 1\\
LTS-966-REQ-0045

}
\hdashrule[0.5ex]{\textwidth}{1pt}{3mm}
  Expected Result \\
{\footnotesize
The tracker's camera is able to be enabled and disabled, returning
ACK300.

}
\hdashrule[0.5ex]{\textwidth}{1pt}{3mm}
  Actual Result \\
{\footnotesize
We got ACK300. Off and On worked fine.~

}
\begin{tabular}{p{2cm}p{14cm}}
\toprule
Step 22 & Step Execution Status: \textbf{ Pass } \\ \hline
\end{tabular}
 Description \\
{\footnotesize
Make a first measurement to get the baseline of the M2 and the Camera
position relative to M1M3

}
\hdashrule[0.5ex]{\textwidth}{1pt}{3mm}
  Expected Result \\
{\footnotesize
The position of the camera in reference to the camera frame is captured.

}
\hdashrule[0.5ex]{\textwidth}{1pt}{3mm}
  Actual Result \\
{\footnotesize
We got ACK 300 and EMP.

}
\begin{tabular}{p{2cm}p{14cm}}
\toprule
Step 23 & Step Execution Status: \textbf{ Pass } \\ \hline
\end{tabular}
 Description \\
{\footnotesize
Move the structure in y axis by about 10mm if possible.\\
Then take the same measurement and record the position. If there is a
ruler available, compare the motion measured with the actual motion.\\
** Then try and bring it back to nominal by moving it back by half of
the value that was measured.\\
Take another measurement **\\[2\baselineskip]Repeat ** ** until we're in
within 1mm.\\[2\baselineskip]\textasciitilde{}5 revolutions is about 1
inch

}
\hdashrule[0.5ex]{\textwidth}{1pt}{3mm}
  Expected Result \\
{\footnotesize
The measurements are taken and are within 1mm as expected.

}
\hdashrule[0.5ex]{\textwidth}{1pt}{3mm}
  Actual Result \\
{\footnotesize
We took an initial measurement using the telescope to verify y = 0mm.
Then we moved the structure to about 10mm using the SA and took another
measurement with the telescope. Using the telescope we confirmed y =
10.05mm. Then again using the values displayed on the SA, we moved the
height of the structure to what was about 5mm. Using the telescope we
confirmed y = 4.995mm. Finally, we moved the structure back to its
initial position and confirmed using the telescope y = 0mm.

}
\begin{tabular}{p{2cm}p{14cm}}
\toprule
Step 24 & Step Execution Status: \textbf{ Pass } \\ \hline
\end{tabular}
 Description \\
{\footnotesize
Move the structure in z axis by less than a 10mm if possible.\\
Then take the same measurement and record the position.\\
** Then try and bring it back to nominal by moving it back by half of
the value that was measured.\\
Take another measurement **\\[2\baselineskip]Repeat ** ** until we're in
within 1mm.\\[2\baselineskip]{[}There is a dial indicator allowing to
move the cart by +/- 0.5 inches. Note that we need to keep track of the
number of revolution of the arrow as each revolution is
\textasciitilde{}0.050 inches\\
.6 inches on the way toward us (12 revolution)\\
.4 inches on the way back{]}\\[2\baselineskip]

}
\hdashrule[0.5ex]{\textwidth}{1pt}{3mm}
  Expected Result \\
{\footnotesize
The measurements are taken and are within 1mm as expected.

}
\hdashrule[0.5ex]{\textwidth}{1pt}{3mm}
  Actual Result \\
{\footnotesize
We initially moved the structure and measured the change to be 38mm
using the telescope. Using the live measurement values displayed by ~SA,
we moved the structure back about halfway. The SA showed that the
structure was now about 14mm, which we confirmed to be the same as
reported by the telescope. ~We moved it back to the 0mm position again
using the SA and confirmed the z was around 0.26mm using the telescope.

}
\begin{tabular}{p{2cm}p{14cm}}
\toprule
Step 25 & Step Execution Status: \textbf{ Pass } \\ \hline
\end{tabular}
 Description \\
{\footnotesize
Rotate the structure along the y axis and take a measurement.\\
** Then try and bring it back to nominal by moving it back by half of
the value that was measured.\\
Take another measurement **\\[2\baselineskip]Repeat ** ** until we're in
within 1mm.

}
\hdashrule[0.5ex]{\textwidth}{1pt}{3mm}
  Expected Result \\
{\footnotesize
The measurements are taken and are within 1mm as expected.~

}
\hdashrule[0.5ex]{\textwidth}{1pt}{3mm}
  Actual Result \\
{\footnotesize
WE were able to measure and query and got the expected results, first
rotating about 11 degrees, then 5.9, then back to within .16 deg of the
original position

}
\begin{tabular}{p{2cm}p{14cm}}
\toprule
Step 26 & Step Execution Status: \textbf{ Pass } \\ \hline
\end{tabular}
 Description \\
{\footnotesize
For the rotation along the z axis, we will simulate the rotation by
swapping Target 1 to Target 2, Target 2 to Target 3 and Target 3 to
Target 1 using the software.\\[2\baselineskip]Then the rotation value
sent toT2SA from the ASC would be 120.\\
Measure the position and verify that the targets in SA have indeed moved
by 120.

}
\hdashrule[0.5ex]{\textwidth}{1pt}{3mm}
  Expected Result \\
{\footnotesize
The measurements are taken and the targets have moved by 120 as
expected.

}
\hdashrule[0.5ex]{\textwidth}{1pt}{3mm}
  Actual Result \\
{\footnotesize
We're simulating camera rotation here without actually altering anything
in the test setup, counter-rotating the expected position in the SA
manually so that we can measure a 120 degree rotation without actually
moving things around. WE also simulated a scenario where the camera
rotator is miscalibrated and we query the camera offsets to measure the
error\\[3\baselineskip]

}
\begin{tabular}{p{2cm}p{14cm}}
\toprule
Step 27 & Step Execution Status: \textbf{ Pass } \\ \hline
\end{tabular}
 Description \\
{\footnotesize
Send the command PUBLISH\_ALT\_AZ\_ROT with ROT = 5deg\\
and repeat the test on the camera

}
\hdashrule[0.5ex]{\textwidth}{1pt}{3mm}
  Expected Result \\
{\footnotesize
The command is accepted and the tracker locates the position of the
SMRs.

}
\hdashrule[0.5ex]{\textwidth}{1pt}{3mm}
  Actual Result \\
{\footnotesize
We got ACK300. When the tracker attempts to measure the test camera
SMRS, we can see it first seeking the 5 degree rotated point, before
finding the real SMR location (since our test rig does not actually
rotate).

}




% This appendix is put in as part of the template. You may edit and add to it.
% It is not overwritten by Docsteady.

\newpage
\appendix
\section{Documentation}
The verification process is defined in \citeds{LSE-160}.
The use of Docsteady to format Jira information in various test and planing documents is
described in \citeds{DMTN-140} and practical commands are given in \citeds{DMTN-178}.

\section{Acronyms used in this document}\label{sec:acronyms}
\input{acronyms.tex}

\newpage

% Uncomment this if Docsteady makes you additional appendix
%\input{SCTR-51.appendix.tex}

\end{document}
